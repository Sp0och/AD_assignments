    \subsection{DONE}
\subsection{DONE}
\subsection{DONE}
\subsection{DONE}
\subsection{}
\subsubsection{DONE}
We fixed the front wheel to remove the singularity of K. $q_{\text{init}}$  was given as:

\begin{equation}\label{eq:4.5.1}
    \begin{split}
        q_{\text{init}} = 
        \begin{pmatrix}
            &\theta_{\text{Frame}} = 0\\
            &x_{\text{Frame}} = 0\\
            &y_{\text{Frame}} = 0.22\\
            &\theta_{\text{Wheel Back}} = 0\\
            % &\theta_{\text{Wheel Front}} = 0\\
            &\theta_{\text{Tire Front}} = 0\\
            &\theta_{\text{Tire Back}} = 0\\
            &y_{\text{Tire Front}} = 0.21\\
            &y_{\text{Tire Back}} = 0.21\\
            &\beta_{\text{Link Back}} = \pi\\
            &\beta_{\text{Link Front}} = 0
        \end{pmatrix}
    \end{split}
\end{equation}

Which represents this position:

\begin{figure}[ht]
    \centering
    \includegraphics[scale=0.235]{images/q_init_1.jpg}
    \caption{Initial Position 1}
    %label always in the end
    \label{fig:init_1}
\end{figure}


This resulted in the equilibrium:

\begin{equation}\label{eq:4.5.2}
    \begin{split}
        q_{\text{equilibrium 1}} = 
        \begin{pmatrix}
            &\theta_{\text{Frame}} = -1.3896e-02 = -0.014\\
            &x_{\text{Frame}} =  -8.1104e-01 = -0.811\\
            &y_{\text{Frame}} = 2.1926e-01 = -0.219\\
            &\theta_{\text{Wheel Back}} = 7.8437e+00 = 7.844\\
            % &\theta_{\text{Wheel Front}} = 0\\
            &\theta_{\text{Tire Front}} = 2.2254e-21 \approx 0\\
            &\theta_{\text{Tire Back}} = 7.8437e+00 = 7.844\\
            &y_{\text{Tire Front}} = 2.1883e-01 = 0.219\\
            &y_{\text{Tire Back}} = 2.1895e-01 = 0.219\\
            &\beta_{\text{Link Back}} = 3.0209e+00 = 3.021\\
            &\beta_{\text{Link Front}} = 1.6787e-01 = 0.168
        \end{pmatrix}
    \end{split}
\end{equation}
Which is visualized by this figure:

\begin{figure}[ht]
    \centering
    \includegraphics[scale=0.235]{images/Equilibrium1.jpg}
    \caption{Equilibrium Position 1}
    %label always in the end
    \label{fig:eq_1}
\end{figure}

\noindent In this example the choice of initial generalized coordinates was obviously very good. For the first task of this assignment the goal is to find two more (probably bad) equilibria.

The eigenvalues of the stiffness matrix are:




\subsubsection{Different Equilibrium states}

The first alternative is luckily already given in the code. Again we fix the front wheel's rotation and start with the initial state:

\begin{equation}\label{eq:4.5.3}
    \begin{split}
        q_{\text{init}} = 
        \begin{pmatrix}
            &\theta_{\text{Frame}} = \pi/2\\
            &x_{\text{Frame}} = 0\\
            &y_{\text{Frame}} = 90\\
            &\theta_{\text{Wheel Back}} = 0\\
            % &\theta_{\text{Wheel Front}} = 0\\
            &\theta_{\text{Tire Front}} = 0\\
            &\theta_{\text{Tire Back}} = 0\\
            &y_{\text{Tire Front}} = 1.90\\
            &y_{\text{Tire Back}} = 0.21\\
            &\beta_{\text{Link Back}} = -\pi/2\\
            &\beta_{\text{Link Front}} = \pi/2
        \end{pmatrix}
    \end{split}
\end{equation}

Which represents this position:


\begin{figure}[ht]
    \centering
    \includegraphics[scale=0.235]{images/q_init_2.jpg}
    \caption{Initial Position 2}
    %label always in the end
    \label{fig:init_2}
\end{figure}
\noindent Note: in the code provided $\theta_{\text{Frame}}$ was $\pi$ which was a different initial set of gen. coord. but converged to the same solution\\
\vspace{2mm}

\noindent As can be seen, this is obviously not a good choice of initial coordinates.

This setting converges to:

\begin{equation}\label{eq:4.5.4}
    \begin{split}
        q_{\text{init}} = 
        \begin{pmatrix}
            &\theta_{\text{Frame}} = 1.5248e+00\\
            &x_{\text{Frame}} = -6.9703e-02\\
            &y_{\text{Frame}} = 1.1280e+00\\
            &\theta_{\text{Wheel Back}} = 3.0427e-01\\
            % &\theta_{\text{Wheel Front}} = 0\\
            &\theta_{\text{Tire Front}} = 2.9622e-24\\
            &\theta_{\text{Tire Back}} = 3.0427e-01\\
            &y_{\text{Tire Front}} = 1.9414e+00\\
            &y_{\text{Tire Back}} = 2.1777e-01\\
            &\beta_{\text{Link Back}} = -1.6327e+00\\
            &\beta_{\text{Link Front}} = 1.5811e+00
        \end{pmatrix}
    \end{split}
\end{equation}

Which looks as follows:

\begin{figure}[ht]
    \centering
    \includegraphics[scale=0.235]{images/Equilibrium2.jpg}
    \caption{Equilibrium Position 2}
    %label always in the end
    \label{fig:eq_2}
\end{figure}

The code example which has even worse initial conditions:

\begin{equation}\label{eq:4.5.5}
    \begin{split}
        q_{\text{init}} = 
        \begin{pmatrix}
            &\theta_{\text{Frame}} = \pi\\
            &x_{\text{Frame}} = 0\\
            &y_{\text{Frame}} = 90\\
            &\theta_{\text{Wheel Back}} = 0\\
            % &\theta_{\text{Wheel Front}} = 0\\
            &\theta_{\text{Tire Front}} = 0\\
            &\theta_{\text{Tire Back}} = 0\\
            &y_{\text{Tire Front}} = 1.90\\
            &y_{\text{Tire Back}} = 0.21\\
            &\beta_{\text{Link Back}} = -\pi/2\\
            &\beta_{\text{Link Front}} = \pi/2
        \end{pmatrix}
    \end{split}
\end{equation}

And looks like this:

\begin{figure}[ht]
    \centering
    \includegraphics[scale=0.235]{images/q_init_3.jpg}
    \caption{Initial Position 3}
    %label always in the end
    \label{fig:init_3}
\end{figure}

Converges to:

\begin{equation}\label{eq:4.5.6}
    \begin{split}
        q_{\text{init}} = 
        \begin{pmatrix}
            &\theta_{\text{Frame}} = 2.0225e+00\\
            &x_{\text{Frame}} =  -1.2734e-01\\
            &y_{\text{Frame}} = 6.3440e-01\\
            &\theta_{\text{Wheel Back}} = 5.5591e-01\\
            % &\theta_{\text{Wheel Front}} = 0\\
            &\theta_{\text{Tire Front}} =  2.0275e-25\\
            &\theta_{\text{Tire Back}} = 5.5591e-01\\
            &y_{\text{Tire Front}} = 1.2043e+00\\
            &y_{\text{Tire Back}} = 2.1777e-01\\
            &\beta_{\text{Link Back}} = -3.4446e+00\\
            &\beta_{\text{Link Front}} = 3.1586e-01
        \end{pmatrix}
    \end{split}
\end{equation}
\clearpage%HERE
Which represents:

\begin{figure}[ht]
    \centering
    \includegraphics[scale=0.235]{images/Equilibrium3.jpg}
    \caption{Equilibrium Position 3}
    %label always in the end
    \label{fig:eq_3}
\end{figure}

\subsection{Analysis}
As we are not considering drag yet we have a conservative, scleronomic system. For an equilibrium in such a system to be stable we need that $\text{K}_{\text{eq}}$ must be positive definite. Aka the eigenvalues of $\text{K}_{\text{eq}}$ must be real and positive.

Looking at the eigenvalues for the first case:
\begin{equation}\label{eq:normal_eigenfrequencies}
    \begin{split}
        \begin{pmatrix}
            1.6913e-07\\
            2.1774e+00\\
            2.2340e+00\\
            3.0009e+01\\
            3.0821e+01\\
            3.4894e+01\\
            3.5642e+01\\
            6.8555e+01\\
            6.8673e+01\\
            1.3035e+02\\
            1.3036e+02
        \end{pmatrix}
    \end{split}
\end{equation}

We can see that they are indeed all positive and real. Thus \ref{eq:4.5.2} is a stable equilibrium. This represents logical assumptions as the car staying in horizontal position is as we know a stable equilibrium.\vspace{3mm}\\

For the second equilibrium we get the following eigenvalues:

\begin{equation}
    \begin{pmatrix}
        0.0000e+00 + 4.9405e-01i\\
        0.0000e+00 + 1.2122e-07i\\
        4.4151e+00 + 0.0000e+00i\\
        7.1552e+00 + 0.0000e+00i\\
        7.4830e+00 + 0.0000e+00i\\
        4.3728e+01 + 0.0000e+00i\\
        4.3983e+01 + 0.0000e+00i\\
        6.3302e+01 + 0.0000e+00i\\
        7.2017e+01 + 0.0000e+00i\\
        7.2124e+01 + 0.0000e+00i\\
        1.2904e+02 + 0.0000e+00i
    \end{pmatrix}
\end{equation}

Where we can see that the eigenvalues ar of complex nature. Mainly the first one

Lastly for the third setting we have the following eigenvalues:

\begin{equation}
    \begin{pmatrix}
        0.0000e+00 + 4.9405e-01i\\
        1.9956e-07 + 0.0000e+00i\\
        4.4151e+00 + 0.0000e+00i\\
        7.1552e+00 + 0.0000e+00i\\
        7.4830e+00 + 0.0000e+00i\\
        4.3728e+01 + 0.0000e+00i\\
        4.3983e+01 + 0.0000e+00i\\
        6.3302e+01 + 0.0000e+00i\\
        7.2017e+01 + 0.0000e+00i\\
        7.2124e+01 + 0.0000e+00i\\
        1.2904e+02 + 0.0000e+00i
    \end{pmatrix}
\end{equation}

Where we see again, that we have complex eigenvalues denoting an unstable equilibrium.

% As we are working with a first order linearization of the dynamics of the system we have to stay in the proximity of the desired state of convergence. So a horizontal car with the initial conditions as seen in equation \ref{eq:4.5.1} will converge to a reasonable solution. \\\vspace{3mm}

% \noindent Initial conditions like a 90° or 180° rotation of the frame can converge to a local solution as seen in Figure \ref*{fig:eq_3}\\\vspace{3mm}

% \noindent The equilibrium from figure \ref{fig:eq_1} surely is stable as we all know it from reality. The two alternatives however are not stable. This can be seen from the example in fig \ref{fig:init_4} when the angle of the frame is between 0 and 90°:
\clearpage %HERE
Another interesting example is the one seen in fig \ref{fig:init_4}:



\begin{figure}[ht]
    \centering
    \includegraphics[scale=0.235]{images/q_init_4.jpg}
    \caption{Initial Position 4}
    %label always in the end
    \label{fig:init_4}
\end{figure}

When we start from the tilted starting position of \ref{fig:init_2} but changing merely the frame angle to a smaller $\pi/4$ we can see that it converges to something quite reasonable looking.
\begin{figure}[ht]
    \centering
    \includegraphics[scale=0.235]{images/Equilibrium4.jpg}
    \caption{Equilibrium Position 4}
    %label always in the end
    \label{fig:eq_4}
\end{figure}

And indeed when looking at the eigenvalues:

\begin{equation}
    \begin{pmatrix}
        2.5694e-07\\
        2.1774e+00\\
        2.2340e+00\\
        3.0009e+01\\
        3.0821e+01\\
        3.4894e+01\\
        3.5642e+01\\
        6.8555e+01\\
        6.8673e+01\\
        1.3035e+02\\
        1.3036e+02
    \end{pmatrix}
\end{equation}

We see that it's a stable equilibrium. From this we can take away that a perturbation from a stable equilibrium that is small enough will converge to the stable equilibrium.   
\subsection{Linearized Equations - DONE}
\subsection{Eigenmodes and Eigenfrequencies}

Looking at the eigenfrequencies in equation (\ref{eq:normal_eigenfrequencies}) we can see that the dominant ones with the large motion impact seem to be the first as well as the second and third.

The eigenmodes associated with these frequencies are:

\begin{equation}
    \begin{split}
        \begin{pmatrix}
            6.0858e-01\\
            9.4719e-02\\
            -9.7389e-02\\
                        0\\
                        0\\
            1.2034e-02\\
            8.8040e-03\\
            -4.7960e-01\\
            5.7699e-01\\
            1.7042e-01\\
            1.3773e-01
        \end{pmatrix}
    \end{split}
\end{equation},

\begin{equation}
    \begin{split}
        \begin{pmatrix}
            4.8632e-02\\
            9.8644e-01\\
                    0\\
                    0\\
                    0\\
            1.0851e-01\\
            1.0851e-01\\
                    0\\
                    0\\
            -1.8772e-02\\
            -2.6059e-02
        \end{pmatrix}
    \end{split}
\end{equation}

and

\begin{equation}
    \begin{split}
        \begin{pmatrix}
            -4.0695e-02\\
                    0\\
            8.0283e-01\\
                    0\\
                    0\\
                    0\\
                    0\\
            -4.0141e-01\\
            -4.0141e-01\\
            -1.2602e-01\\
            1.2515e-01
        \end{pmatrix}
    \end{split}
\end{equation}

Plotted out we get the following configurations:

\begin{figure}[ht]
    \centering
    \includegraphics[scale=0.235]{images/mode1.jpg}
    \caption{Eigenmode Configuration 1}
    %label always in the end
    \label{fig:mode_1}
\end{figure}

\begin{figure}[ht]
    \centering
    \includegraphics[scale=0.235]{images/mode2.jpg}
    \caption{Eigenmode Configuration 2}
    %label always in the end
    \label{fig:mode_2}
\end{figure}

\begin{figure}[ht]
    \centering
    \includegraphics[scale=0.235]{images/mode3.jpg}
    \caption{Eigenmode Configuration 3}
    %label always in the end
    \label{fig:mode_3}
\end{figure}

This result I am not sure about at the current time.

\subsubsection{DONE}
\subsubsection{Mode Shapes}

\subsection{Drag and Damping}
\subsubsection{Damping}
The contributions to the damping force are:
\begin{equation}
    \begin{split}
        Q_{\text{suspension back}} = \begin{pmatrix}
            Q_{\text{susb(1)}}\\
            0\\
            0\\
            0\\
            0\\
            0\\
            0\\
            0\\
            0\\
            Q_{\text{susb(2)}}\\
            0
        \end{pmatrix}
    \end{split}
\end{equation}
Which has a contribution towards the rotation of the frame and the rotation of the back link. Both of these contributions have a negative sign which makes sense as the suspension of the back tire will lift the back wheel resulting in a rotation of the bar and the frame in negative Z direction.

Analogous for the front suspension:
\begin{equation}
    \begin{split}
        Q_{\text{suspension front}} = \begin{pmatrix}
            Q_{\text{susf(1)}}\\
            0\\
            0\\
            0\\
            0\\
            0\\
            0\\
            0\\
            0\\
            0\\
            Q_{\text{susf(2)}}
        \end{pmatrix}
    \end{split}
\end{equation}
Therefore the front suspension force leads to an impact on the frame angle and the front link angle.

Note: these two suspension forces come from the spring approximation of the reality.

\begin{equation}
    \begin{split}
        Q_\text{tire back} = \begin{pmatrix}
            0\\
            0\\
            0\\
            0\\
            0\\
            0\\
            0\\
            Q_\text{tB(9)}\\
            0\\
            0\\
            0
        \end{pmatrix}
    \end{split}
\end{equation}
The back tire contact force has an impact on the elevation of the back tire.

\begin{equation}
    \begin{split}
        Q_\text{tire front} = \begin{pmatrix}
            0\\
            0\\
            0\\
            0\\
            0\\
            0\\
            0\\
            Q_\text{tF(9)}\\
            0\\
            0\\
            0
        \end{pmatrix}
    \end{split}
\end{equation}
This has an impact on the elevation of the front tire component.
\subsubsection{Drag Forces}
\begin{equation}
    \begin{split}
        Q_\text{drag} =
        \begin{pmatrix}
            0\\
-(63*\dot x_\text{fr}^2)/100\\
                0\\
                0\\
                0\\
                0\\
                0\\
                0\\
                0\\
                0\\
                0
        \end{pmatrix}
    \end{split}
\end{equation}
This is the aerodynamic drag on the system. This can be nicely seen as there is only a negative component in the second row (depending on the velocity in this direction scquared) which corresponds to the x movement of the frame. Thus we have aerodynamic drag in that direction.

\begin{equation}Q_\text{friction wheels} = 
    \begin{pmatrix}
        0\\
        0\\
        0\\
-\dot\theta_\text{wF}^3/50000\\
-\dot\theta_\text{wB}^3/50000\\
        0\\
        0\\
        0\\
        0\\
        0\\
        0
    \end{pmatrix}
\end{equation}
Here we can nicely see friciton drag impacting the rotation of the front and the back wheel.

Lastly we have for the torque:
\begin{equation}
    Q_\text{wheel} = 
    \begin{pmatrix}
        0\\
        0\\
        0\\
        0\\
        -300\\
        0\\
        0\\
        0\\
        0\\
        0\\
        0
    \end{pmatrix}
\end{equation}
Representing the load applied in negative Z direction on the back wheel to make the car drive to the right.

\subsection{Nonlinear Time Integration}
\textbf{Speed of the Car:}
\begin{figure}[ht]
    \centering
    \includegraphics[scale=0.4]{images/convergence_speed_of_car.jpg}
    \caption{Speed of Car with convergence speed indicated}
    %label always in the end
    \label{fig:convergence_speed}
\end{figure}

As we can see in figure \ref{fig:convergence_speed} the horizontal velocity converges towards 33.64 m/s which is equivalent to 120.6 km/h. (When performing the integration for a longer time interval the more exact convergence value is 33.6466) Also for an acceleration to 80 km/h ($= 22.22 $ m/s) it takes about 2.8s.
\clearpage %HERE

\subsubsection{Horizontal displacement of the frame}
\begin{figure}[ht]
    \centering
    \includegraphics[scale=0.4]{images/horizontal_displacement.jpg}
    \caption{Horizontal Displacement}
    %label always in the end
    \label{fig:horizontal_displacement}
\end{figure}

Here above we can nicely the horizontal displacement with time. As already assumed the displacement starts at zero and at one point behaves linear in time due to the constant velocity which can be seen in figure \ref{fig:convergence_speed}.
\clearpage %HERE

\subsubsection{Vertical displacement of the frame}
\begin{figure}[ht]
    \centering
    \includegraphics[scale=0.4]{images/vertical_displacement.jpg}
    \caption{Vertical Displacement}
    %label always in the end
    \label{fig:vertical_displacement}
\end{figure}
Here we can see a very small oscillating displacement which converges towards 0.
\clearpage %HERE

\subsubsection{Wheel-ground elastic forces}
As the wheel-ground elastic forces are modelled with linear springs we have:
\begin{equation}
    \begin{split}
        &f_\text{Contact} = k_\text{Contact} * \Delta y_\text{tF}\\
        & \text{ with } \Delta y_\text{tF} = |y_\text{tF}-r_\text{tire}|
    \end{split}
\end{equation}
And thus for the elastic forces of the tire we get the following plot:
\begin{figure}[ht]
    \centering
    \includegraphics[scale=0.4]{images/wheel-ground_elastic_forces.jpg}
    \caption{Elastic Contact Forces}
    %label always in the end
    \label{fig:contact_forces}
\end{figure}

\clearpage %HERE
\subsubsection{orientation of the rear linkage}
\begin{figure}[ht]
    \centering
    \includegraphics[scale=0.4]{images/rear_linkage_angle.jpg}
    \caption{Rear Linkage angle}
    %label always in the end
    \label{fig:rear_linkage_angle}
\end{figure}
\subsection{Adding Wings}
To add the effect of the wings to the car we have to derive the contribution of the generalized forces. For this we calculate the work done at either wing and consider the derivate w.r.t. q. 
\begin{equation}
    \begin{split}
        Q_\text{L Front} = \frac{\partial}{\partial q}W_\text{L Front}
    \end{split}
\end{equation}
For the work done we multiply the force with the respective velocity component. 

The velocity component consists of both ends of the rigid body translation and the contribution from the frame rotation.
\begin{equation}
    \begin{split}
        v_F = v_\text{Frame} + \begin{pmatrix}
            0\\0\\\dot\theta
        \end{pmatrix} \times \begin{pmatrix}
            x_F\\
            y_F\\
            0
        \end{pmatrix}
    \end{split}
\end{equation}
\begin{equation}
    \begin{split}
        \Rightarrow W_\text{L Front} = F_\text{L Front} * v_F
    \end{split}
\end{equation}
\subsubsection{Horizontal Displacement of the Frame}   
Implementing the above contribution of the wings in the code yields the following plots: 

\begin{figure}[ht]
    \centering
    \includegraphics[scale=0.4]{images/horizontal_displacement_wings.jpg}
    \caption{Horizontal Displacement with Wings}
    %label always in the end
    \label{fig:horizontal_wings}
\end{figure}
The distance covered as can be seen above is significantly less than for the case without wings. This is due to the drag induced by the wings.
\clearpage %HERE
\subsubsection{Vertical Displacement of the Frame}
\begin{figure}[ht]
    \centering
    \includegraphics[scale=0.4]{images/vertical_displacement_wings.jpg}
    \caption{Vertical Displacement with Wings}
    %label always in the end
    \label{fig:vertical_wings}
\end{figure}
The vertical displacement is stable at first and then the car gains altitude rapidly. As the wings induce drag forces the car seems to tip over at the front resulting in an elevation of the center of mass.
\clearpage %HERE
\subsubsection{Wheel-Ground Elastic Forces}
\begin{figure}[ht]
    \centering
    \includegraphics[scale=0.4]{images/wheel-ground_elastic_forces_wings.jpg}
    \caption{Elastic Contact Forces with Wings}
    %label always in the end
    \label{fig:elastic_forces_wings}
\end{figure}
Starting off we see the elastic force in the same range as without the wings. Around the 5 second mark the elastic force increases by a lot. This is the same point in time as the aforementioned supposed tipping of the car.
\clearpage %HERE
\subsubsection{Orientation of the Rear Linkage}
\begin{figure}[ht]
    \centering
    \includegraphics[scale=0.35]{images/rear_linkage_angle_with_wings.jpg}
    \caption{Rear Linkage Orientation with Wings}
    %label always in the end
    \label{fig:rear_linkage_wings}
\end{figure}
The rear linkage angle starts off at 180° as before but then turns clockwise.
\clearpage
\textbf{Wings Conclusion}
\begin{figure}[ht]
    \centering
    \includegraphics[scale=0.35]{images/end_position_wings.jpg}
    \caption{End Pose Car with Wings}
    %label always in the end
    \label{fig:end_position_wings}
\end{figure}
Looking at the last state of the generalized coordinates we can see the car in a tilted state. From the additional drag force from the wings the car seems to turn in a clockwise motion.
\subsection{Harmonic Response}
Not completed.

