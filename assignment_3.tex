A few notes before starting this assignment:

I'm considering three frames. 

The first one is the inertial frame. It is in A with the first axis straight up, the second right and the third into the plane of the image.

\begin{figure}[ht]
    \centering
    \includegraphics[scale=0.5]{images/inertial_frame.png}
    \caption{Inertial Frame}
    %label always in the end
    \label{fig:inertial_frame}
\end{figure}

\noindent The second one is located at the same position but rotating with Omega. I will call it the vertical frame.

\begin{figure}[ht]
    \centering
    \includegraphics[scale=0.5]{images/vertical_frame.png}
    \caption{Vertical Frame}
    %label always in the end
    \label{fig:vertical_frame}
\end{figure}
\clearpage %HERE

The third one that is called rod frame is the same as the vertical frame but with a constant rotation of $\beta$ around the negative $e3_v$ axis.

\begin{figure}[ht]
    \centering
    \includegraphics[scale=0.5]{images/rod_frame.png}
    \caption{Rod Frame}
    %label always in the end
    \label{fig:rod_frame}
\end{figure}

The last frame is the square frame which relative to the rod frame is also rotation with $\phi$:

\begin{figure}[ht]
    \centering
    \includegraphics[scale=0.5]{images/square_frame.png}
    \caption{Square Frame}
    %label always in the end
    \label{fig:square_frame}
\end{figure}


\subsection{Kinetic and Potential Energy}
\subsubsection{Vertical Shaft}
As the vertical bar rotates around the thing axis and is static regarding the height of it's CoM the contribution to the energ is 0.

\begin{equation}
    \begin{split}
        T_\text{vertical bar} = V_\text{vertical bar} = 0
    \end{split}
\end{equation}
\subsubsection{Horizontal Bar}
\textbf{Note: } I will refer to the horizontal bar as bar and to the tilted bar as rod.

As point A which is part of the bar is not moving we can simply consider the rotational part w.r.t. to this fixed point A.:

\begin{equation}
    \begin{split}
        T_\text{bar} =  \frac{L^2\Omega ^2m}{6}
    \end{split}
\end{equation}

As the bar is horizontal on the height of A we have no contribution of the potential energy.

\begin{equation}
    \begin{split}
        V_\text{bar} = 0
    \end{split}
\end{equation}

\subsubsection{Rod}
As the rod has no static point we will use a superposition of the rotational and translational kinetic energy.

\begin{equation}
    \begin{split}
        T_\text{Rod transl} = \frac{L^2\Omega ^2m{\left(\cos\left(\beta \right)+2\right)}^2}{8}
    \end{split}
\end{equation}
For the rotational part considering the moment of inertia of the center of mass we get:
\begin{equation}
    \begin{split}
        T_\text{Rod rot} = \frac{L^2\Omega ^2m{\cos\left(\beta \right)}^2}{24}
    \end{split}
\end{equation}
Which leads to:
\begin{equation}
    \begin{split}
        \Rightarrow T_\text{Rod} = \frac{L^2\Omega ^2m\left({\cos\left(\beta \right)}^2+3\cos\left(\beta \right)+3\right)}{6}
    \end{split}
\end{equation}

For the potential energy we have the verical part of the incline of the rod:

\begin{equation}
    \begin{split}
        V_\text{Rod} = \frac{L}{2}mg\sin\beta
    \end{split}
\end{equation}

\subsubsection{Square Plate}
As for the rod, first the translational kinetic energy:

\begin{equation}
    \begin{split}
        T_\text{square transl} = \frac{M\left({\left(\left(\Omega +\mathrm{\dot\phi}\sin\left(\beta \right)\right)\left(\frac{L\left(\sin\left(\Omega t\right)\sin\left(\phi \right)-\cos\left(\Omega t\right)\cos\left(\phi \right)\sin\left(\beta \right)\right)}{2}-\frac{L\cos\left(\Omega t\right)\cos\left(\beta \right)}{2}\right)-L\Omega \cos\left(\Omega t\right)+\frac{L\mathrm{\dot\phi}\cos\left(\Omega t\right)\cos\left(\beta \right)\left(\sin\left(\beta \right)-\cos\left(\beta \right)\cos\left(\phi \right)\right)}{2}\right)}^2+{\left(\left(\Omega +\mathrm{\dot\phi}\sin\left(\beta \right)\right)\left(\frac{L\left(\cos\left(\Omega t\right)\sin\left(\phi \right)+\sin\left(\Omega t\right)\cos\left(\phi \right)\sin\left(\beta \right)\right)}{2}+\frac{L\sin\left(\Omega t\right)\cos\left(\beta \right)}{2}\right)+L\Omega \sin\left(\Omega t\right)-\frac{L\mathrm{\dot\phi}\sin\left(\Omega t\right)\cos\left(\beta \right)\left(\sin\left(\beta \right)-\cos\left(\beta \right)\cos\left(\phi \right)\right)}{2}\right)}^2+\frac{L^2{\mathrm{\dot\phi}}^2{\cos\left(\beta \right)}^2{\sin\left(\phi \right)}^2}{4}\right)}{2}
    \end{split}
\end{equation}

For the rotational part we get considering the moment of inertia w.r.t. the c.o.m. in the s frame:

\begin{equation}
    \begin{split}
        T_\text{square rot} = \frac{L^2M\left(\mathrm{\dot\phi}+\Omega \sin\left(\beta \right)\right)\left(\frac{\mathrm{\dot\phi}}{2}+\frac{\Omega \sin\left(\beta \right)}{2}\right)}{12}+\frac{L^2M\Omega ^2{\cos\left(\beta \right)}^2{\cos\left(\phi \right)}^2}{24}
    \end{split}
\end{equation}

Which leads to:

\begin{equation}
    \begin{split}
        T_\text{square} = \frac{L^2M\left(-\Omega ^2{\cos\left(\beta \right)}^2{\cos\left(\phi \right)}^2+\Omega ^2{\cos\left(\beta \right)}^2+3\sin\left(\beta \right)\Omega ^2\cos\left(\beta \right)\cos\left(\phi \right)+6\Omega ^2\cos\left(\beta \right)+6\sin\left(\beta \right)\Omega ^2\cos\left(\phi \right)+8\Omega ^2+3\Omega \mathrm{\dot\phi}\cos\left(\beta \right)\cos\left(\phi \right)+6\Omega \mathrm{\dot\phi}\cos\left(\phi \right)+4\sin\left(\beta \right)\Omega \mathrm{\dot\phi}+2{\mathrm{\dot\phi}}^2\right)}{12}
    \end{split}
\end{equation}

For the potential energy we get:
\begin{equation}
    \begin{split}
        \frac{LMg\left(\sin\left(\beta \right)-\cos\left(\beta \right)\cos\left(\phi \right)\right)}{2}
    \end{split}
\end{equation}

\subsubsection{Total Energy}
The total kinetic energy comes to:

\begin{equation}
    \begin{split}
        &T = \underbrace{T_\text{vertical bar}}_{0} + T_\text{bar} + T_\text{Rod} + T_\text{Square} = \\
        & \frac{L^2\left(8M\Omega ^2+8\Omega ^2m+2M{\mathrm{\dot\phi}}^2+6M\Omega ^2\cos\left(\beta \right)+6\Omega ^2m\cos\left(\beta \right)+M\Omega ^2{\cos\left(\beta \right)}^2+2\Omega ^2m{\cos\left(\beta \right)}^2+6M\Omega ^2\cos\left(\phi \right)\sin\left(\beta \right)+6M\Omega \mathrm{\dot\phi}\cos\left(\phi \right)+4M\Omega \mathrm{\dot\phi}\sin\left(\beta \right)-M\Omega ^2{\cos\left(\beta \right)}^2{\cos\left(\phi \right)}^2+3M\Omega \mathrm{\dot\phi}\cos\left(\beta \right)\cos\left(\phi \right)+3M\Omega ^2\cos\left(\beta \right)\cos\left(\phi \right)\sin\left(\beta \right)\right)}{12}
    \end{split}
\end{equation}

Analogous the potential energy:
\begin{equation}
    \begin{split}
        &V = \underbrace{V_\text{vertical bar}}_{0} + V_\text{bar} + V_\text{Rod} + V_\text{Square} = \\
        & \frac{LMg\left(2\sin\left(\beta \right)-\cos\left(\beta \right)\cos\left(\phi \right)\right)}{2}
    \end{split}
\end{equation}


\subsection{Lagrange Equations}
Having an expression for the kinetic and potential energy we can use them to denote our lagrange equations:

\begin{equation}
    \begin{split}
        \frac{\partial}{\partial t}\frac{\partial T}{\partial \dot\phi} - \frac{\partial T}{\partial \phi} + \frac{\partial V}{\partial \phi} = 0
    \end{split}
\end{equation}

Which yields:

\begin{equation}
    \begin{split}
        \frac{LM\left(-2L\cos\left(\phi \right)\sin\left(\phi \right)\Omega ^2{\cos\left(\beta \right)}^2+3L\sin\left(\beta \right)\sin\left(\phi \right)\Omega ^2\cos\left(\beta \right)+6L\sin\left(\beta \right)\sin\left(\phi \right)\Omega ^2+6g\sin\left(\phi \right)\cos\left(\beta \right)+4L\mathrm{\ddot\phi}\right)}{12} = 0
    \end{split}
\end{equation}

Which is a differential equation for $\phi$.

\subsection{Non constant angular velocity around vertical Shaft}
\subsubsection{Vertical shaft}
Unchanged
\subsubsection{Bar}

 \begin{equation}
    \begin{split}
        T_\text{bar} = \frac{L^2\Omega ^2m}{6}
    \end{split}
 \end{equation}

 \begin{equation}
    \begin{split}
        V_\text{bar} = 0
    \end{split}
 \end{equation}

 \subsubsection{Rod}

 \begin{equation}
    \begin{split}
        T_\text{rod} = \frac{L^2m{\mathrm{\dot\theta}}^2\left({\cos\left(\beta \right)}^2+3\cos\left(\beta \right)+3\right)}{6}
    \end{split}
 \end{equation}

 \begin{equation}
    \begin{split}
        V_\text{rod} = \frac{Lgm\sin\left(\beta \right)}{2}
    \end{split}
 \end{equation}

 \subsubsection{Square}

 \begin{equation}
    \begin{split}
        T_\text{square} = \frac{L^2M\left(2{\mathrm{\dot\phi}}^2+3\mathrm{\dot\phi}\mathrm{\dot\theta}\cos\left(\beta \right)\cos\left(\phi \right)+6\mathrm{\dot\phi}\mathrm{\dot\theta}\cos\left(\phi \right)+4\sin\left(\beta \right)\mathrm{\dot\phi}\mathrm{\dot\theta}-{\mathrm{\dot\theta}}^2{\cos\left(\beta \right)}^2{\cos\left(\phi \right)}^2+{\mathrm{\dot\theta}}^2{\cos\left(\beta \right)}^2+3\sin\left(\beta \right){\mathrm{\dot\theta}}^2\cos\left(\beta \right)\cos\left(\phi \right)+6{\mathrm{\dot\theta}}^2\cos\left(\beta \right)+6\sin\left(\beta \right){\mathrm{\dot\theta}}^2\cos\left(\phi \right)+8{\mathrm{\dot\theta}}^2\right)}{12}
    \end{split}
 \end{equation}

 \begin{equation}
    \begin{split}
        V_\text{square} = \frac{LMg\left(\sin\left(\beta \right)-\cos\left(\beta \right)\cos\left(\phi \right)\right)}{2}
    \end{split}
 \end{equation}

 \subsubsection{Total Energy}

 \begin{equation}
    \begin{split}
        T = \frac{L^2\left(2M{\mathrm{\dot\phi}}^2+8M{\mathrm{\dot\theta}}^2+8m{\mathrm{\dot\theta}}^2+6M{\mathrm{\dot\theta}}^2\cos\left(\beta \right)+6m{\mathrm{\dot\theta}}^2\cos\left(\beta \right)+M{\mathrm{\dot\theta}}^2{\cos\left(\beta \right)}^2+2m{\mathrm{\dot\theta}}^2{\cos\left(\beta \right)}^2+6M{\mathrm{\dot\theta}}^2\cos\left(\phi \right)\sin\left(\beta \right)+6M\mathrm{\dot\phi}\mathrm{\dot\theta}\cos\left(\phi \right)+4M\mathrm{\dot\phi}\mathrm{\dot\theta}\sin\left(\beta \right)-M{\mathrm{\dot\theta}}^2{\cos\left(\beta \right)}^2{\cos\left(\phi \right)}^2+3M\mathrm{\dot\phi}\mathrm{\dot\theta}\cos\left(\beta \right)\cos\left(\phi \right)+3M{\mathrm{\dot\theta}}^2\cos\left(\beta \right)\cos\left(\phi \right)\sin\left(\beta \right)\right)}{12}
    \end{split}
 \end{equation}

 \begin{equation}
    \begin{split}
        V = \frac{Lgm\sin\left(\beta \right)}{2}+\frac{LMg\left(\sin\left(\beta \right)-\cos\left(\beta \right)\cos\left(\phi \right)\right)}{2}
    \end{split}
 \end{equation}

 \subsubsection{Lagrange Equations}
 As now we have two generalized coordinates we also have two lagrangr equations. For $\phi$ we get:

 \begin{equation}
    \begin{split}
        \frac{LM\left(-2L\cos\left(\phi \right)\sin\left(\phi \right){\mathrm{\dot\theta}}^2{\cos\left(\beta \right)}^2+3L\sin\left(\beta \right)\sin\left(\phi \right){\mathrm{\dot\theta}}^2\cos\left(\beta \right)+6L\sin\left(\beta \right)\sin\left(\phi \right){\mathrm{\dot\theta}}^2+6g\sin\left(\phi \right)\cos\left(\beta \right)+4L\mathrm{\ddot\phi}\right)}{12}
    \end{split}
 \end{equation}

 For $\theta$:

 \begin{equation}
    \begin{split}
        \frac{L^2\mathrm{\ddot\theta}\left(16M+16m+12M\cos\left(\beta \right)+12m\cos\left(\beta \right)+2M{\cos\left(\beta \right)}^2+4m{\cos\left(\beta \right)}^2-2M{\cos\left(\beta \right)}^2{\cos\left(\phi \right)}^2+12M\cos\left(\phi \right)\sin\left(\beta \right)+6M\cos\left(\beta \right)\cos\left(\phi \right)\sin\left(\beta \right)\right)}{12}
    \end{split}
 \end{equation}


\subsection{Non Conservative Forces}
Now we have a non conservative force attacking on the square.

We have to derive the velocity of a random point on the square. In the following I will parametrize said point with $x_1$ and $x_2$ which are the coordinates of the point on the square in the s frame from B.

\begin{equation}
    \begin{split}
        v_{Pi} &= v_{Bi} + \omega_i \times BP_i\\
        &=\begin{pmatrix}
            &-\mathrm{\dot\phi}x_{1}\cos\left(\beta \right)\sin\left(\phi \right)\\
            &\left(x_{1}\left(\cos\left(\theta \right)\sin\left(\phi \right)+\cos\left(\phi \right)\sin\left(\beta \right)\sin\left(\theta \right)\right)-x_{2}\cos\left(\beta \right)\sin\left(\theta \right)\right)\left(\mathrm{\dot\theta}+\mathrm{\dot\phi}\sin\left(\beta \right)\right)-L\mathrm{\dot\theta}\sin\left(\theta \right)+\mathrm{\dot\phi}\cos\left(\beta \right)\sin\left(\theta \right)\left(x_{2}\sin\left(\beta \right)+x_{1}\cos\left(\beta \right)\cos\left(\phi \right)\right)\\
            &\left(x_{1}\left(\sin\left(\phi \right)\sin\left(\theta \right)-\cos\left(\phi \right)\sin\left(\beta \right)\cos\left(\theta \right)\right)+x_{2}\cos\left(\beta \right)\cos\left(\theta \right)\right)\left(\mathrm{\dot\theta}+\mathrm{\dot\phi}\sin\left(\beta \right)\right)+L\mathrm{\dot\theta}\cos\left(\theta \right)-\mathrm{\dot\phi}\cos\left(\beta \right)\cos\left(\theta \right)\left(x_{2}\sin\left(\beta \right)+x_{1}\cos\left(\beta \right)\cos\left(\phi \right)\right)
        \end{pmatrix}
    \end{split}
\end{equation}

Therefore the work of the force contribution at a single point P is:

\begin{equation}
    \begin{split}
        dW &= -cvP_i\bullet vP_i = \\
        &= -c{\left(\left(x_{1}\left(\sin\left(\phi \right)\sin\left(\theta \right)-\cos\left(\phi \right)\sin\left(\beta \right)\cos\left(\theta \right)\right)+x_{2}\cos\left(\beta \right)\cos\left(\theta \right)\right)\left(\mathrm{\dot\theta}+\mathrm{\dot\phi}\sin\left(\beta \right)\right)+L\mathrm{\dot\theta}\cos\left(\theta \right)-\mathrm{\dot\phi}\cos\left(\beta \right)\cos\left(\theta \right)\left(x_{2}\sin\left(\beta \right)+x_{1}\cos\left(\beta \right)\cos\left(\phi \right)\right)\right)}^2-c{\left(\left(x_{1}\left(\cos\left(\theta \right)\sin\left(\phi \right)+\cos\left(\phi \right)\sin\left(\beta \right)\sin\left(\theta \right)\right)-x_{2}\cos\left(\beta \right)\sin\left(\theta \right)\right)\left(\mathrm{\dot\theta}+\mathrm{\dot\phi}\sin\left(\beta \right)\right)-L\mathrm{\dot\theta}\sin\left(\theta \right)+\mathrm{\dot\phi}\cos\left(\beta \right)\sin\left(\theta \right)\left(x_{2}\sin\left(\beta \right)+x_{1}\cos\left(\beta \right)\cos\left(\phi \right)\right)\right)}^2-c{\mathrm{\dot\phi}}^2{x_{1}}^2{\cos\left(\beta \right)}^2{\sin\left(\phi \right)}^2
    \end{split}
\end{equation}

As we want to derive the whole work done by the area distributed force we integrate the expression of dW over the whole square:

\begin{equation}
    \begin{split}
        W = \int_{x_1 = -L}^0\int_{x_2 = 0}^L-c{\left(\left(x_{1}\left(\sin\left(\phi \right)\sin\left(\theta \right)-\cos\left(\phi \right)\sin\left(\beta \right)\cos\left(\theta \right)\right)+x_{2}\cos\left(\beta \right)\cos\left(\theta \right)\right)\left(\mathrm{\dot\theta}+\mathrm{\dot\phi}\sin\left(\beta \right)\right)+L\mathrm{\dot\theta}\cos\left(\theta \right)-\mathrm{\dot\phi}\cos\left(\beta \right)\cos\left(\theta \right)\left(x_{2}\sin\left(\beta \right)+x_{1}\cos\left(\beta \right)\cos\left(\phi \right)\right)\right)}^2-c{\left(\left(x_{1}\left(\cos\left(\theta \right)\sin\left(\phi \right)+\cos\left(\phi \right)\sin\left(\beta \right)\sin\left(\theta \right)\right)-x_{2}\cos\left(\beta \right)\sin\left(\theta \right)\right)\left(\mathrm{\dot\theta}+\mathrm{\dot\phi}\sin\left(\beta \right)\right)-L\mathrm{\dot\theta}\sin\left(\theta \right)+\mathrm{\dot\phi}\cos\left(\beta \right)\sin\left(\theta \right)\left(x_{2}\sin\left(\beta \right)+x_{1}\cos\left(\beta \right)\cos\left(\phi \right)\right)\right)}^2-c{\mathrm{\dot\phi}}^2{x_{1}}^2{\cos\left(\beta \right)}^2{\sin\left(\phi \right)}^2dx_2dx_1
    \end{split}
\end{equation}

Which comes to:

\begin{equation}
    \begin{split}
        W = -\frac{cL^4{\mathrm{\dot\phi}}^2}{3}-\frac{cL^4\mathrm{\dot\phi}\mathrm{\dot\theta}\cos\left(\beta \right)\cos\left(\phi \right)}{2}-cL^4\mathrm{\dot\phi}\mathrm{\dot\theta}\cos\left(\phi \right)-\frac{2c\sin\left(\beta \right)L^4\mathrm{\dot\phi}\mathrm{\dot\theta}}{3}+\frac{cL^4{\mathrm{\dot\theta}}^2{\cos\left(\beta \right)}^2{\cos\left(\phi \right)}^2}{3}-\frac{cL^4{\mathrm{\dot\theta}}^2{\cos\left(\beta \right)}^2}{3}-\frac{c\sin\left(\beta \right)L^4{\mathrm{\dot\theta}}^2\cos\left(\beta \right)\cos\left(\phi \right)}{2}-cL^4{\mathrm{\dot\theta}}^2\cos\left(\beta \right)-c\sin\left(\beta \right)L^4{\mathrm{\dot\theta}}^2\cos\left(\phi \right)-\frac{4cL^4{\mathrm{\dot\theta}}^2}{3}
    \end{split}
\end{equation}

And finally the generalized forces are:

\begin{equation}
    \begin{split}
        Q_\phi = \frac{\partial W}{\partial \phi} = -\frac{L^4c\left(4\mathrm{\dot\phi}+6\mathrm{\dot\theta}\cos\left(\phi \right)+4\mathrm{\dot\theta}\sin\left(\beta \right)+3\mathrm{\dot\theta}\cos\left(\beta \right)\cos\left(\phi \right)\right)}{6}
    \end{split}
\end{equation}

And 

\begin{equation}
    \begin{split}
        Q_\theta = -\frac{L^4c\left(16\mathrm{\dot\theta}+12\mathrm{\dot\theta}\cos\left(\beta \right)+6\mathrm{\dot\phi}\cos\left(\phi \right)+4\mathrm{\dot\phi}\sin\left(\beta \right)+4\mathrm{\dot\theta}{\cos\left(\beta \right)}^2-4\mathrm{\dot\theta}{\cos\left(\beta \right)}^2{\cos\left(\phi \right)}^2+3\mathrm{\dot\phi}\cos\left(\beta \right)\cos\left(\phi \right)+12\mathrm{\dot\theta}\cos\left(\phi \right)\sin\left(\beta \right)+6\mathrm{\dot\theta}\cos\left(\beta \right)\cos\left(\phi \right)\sin\left(\beta \right)\right)}{6}
    \end{split}
\end{equation}