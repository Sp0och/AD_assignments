\subsection{}
Lagrange equations:
\begin{equation}
    \begin{split}
        &\frac{L^2M\mathrm{\ddot{\phi}}}{3}+\frac{L^2m\mathrm{\ddot{\phi}}}{3}-\frac{L^2m\mathrm{\ddot{\phi}}{\cos\left(\beta \right)}^2}{3}+\frac{LMg\cos\left(\beta \right)\sin\left(\phi \right)}{2}+\frac{L^2M\Omega ^2\sin\left(\beta \right)\sin\left(\phi \right)}{2}+\\
        &\frac{L^2M\Omega ^2\cos\left(\beta \right)\sin\left(\beta \right)\sin\left(\phi \right)}{4}-\frac{L^2M\Omega ^2{\cos\left(\beta \right)}^2\cos\left(\phi \right)\sin\left(\phi \right)}{6} = 0
    \end{split}
\end{equation}
\subsection{}
This can be reformulated for the differential equation of $\ddot{\phi}$:

\begin{equation}
    \begin{split}
        &\ddot{\phi} \left(\frac{L^2(M+m)}{3} - \frac{L^2m\cos(\beta)^2}{3}\right) + \sin(\phi)\left(\frac{LMg\cos(\beta)}{2} + \frac{L^2M\Omega^2\sin(\beta)}{2} + \frac{L^2M\Omega^2\cos(\beta)\sin(\beta)}{4}\right) - \\
        &\frac{L^2M\Omega^2\cos(\beta)^2\cos(\phi)\sin(\phi)}{6} = 0
    \end{split}
\end{equation}

\subsection{}
Equation of motion for the case that the rotation around the vertical bar is not constant:

\begin{equation}
    \begin{split}
        &\frac{L^2\mathrm{\ddot{\theta}}\left(16M+16m+12M\cos\left(\beta \right)+12m\cos\left(\beta \right)+2M{\cos\left(\beta \right)}^2+4m{\cos\left(\beta \right)}^2-2M{\cos\left(\beta \right)}^2{\cos\left(\phi \right)}^2\right)}{12}+\\
        &\frac{L^2\mathrm{\ddot{\theta}}\left(12M\cos\left(\phi \right)\sin\left(\beta \right)+6M\cos\left(\beta \right)\cos\left(\phi \right)\sin\left(\beta \right)\right)}{12}
    \end{split}
\end{equation}


\subsection{}
\begin{equation}
    \begin{split}
        &-c{\left(L\mathrm{\dot{\theta}}+\mathrm{\dot{\theta}}x_{2}\cos\left(\beta \right)-\mathrm{\dot{\phi}}x_{1}\cos\left(\phi \right)+L\mathrm{\dot{\theta}}\cos\left(\beta \right)-\mathrm{\dot{\theta}}x_{1}\cos\left(\phi \right)\sin\left(\beta \right)\right)}^2-\\
        &c{x_{1}}^2{\sin\left(\phi \right)}^2{\left(\mathrm{\dot{\theta}}+\mathrm{\dot{\phi}}\sin\left(\beta \right)\right)}^2-c{\mathrm{\dot{\phi}}}^2{x_{1}}^2{\cos\left(\beta \right)}^2{\sin\left(\phi \right)}^2 
    \end{split}
\end{equation}

To find the whole work done by this force distribution we have to integrate it:

$x_1$ goes from -L to 0 and $x_2$ goes from -L/2 to L/2.

\begin{equation}
    \begin{split}
        &-c\int_{-L}^0\int_{-\frac{L}{2}}^{\frac{L}{2}}
        \underbrace{\left[{\left(L\mathrm{\dot{\theta}}+\mathrm{\dot{\theta}}x_{2}\cos\left(\beta \right)-\mathrm{\dot{\phi}}x_{1}\cos\left(\phi \right)+L\mathrm{\dot{\theta}}\cos\left(\beta \right)-\mathrm{\dot{\theta}}x_{1}\cos\left(\phi \right)\sin\left(\beta \right)\right)}^2\right]}_{\text{A}}
        dx_2dx_1\\
        &-c\int_{-L}^0\int_{-\frac{L}{2}}^{\frac{L}{2}}
        \underbrace{\left[{x_{1}}^2\left({\sin\left(\phi \right)}^2{\left(\mathrm{\dot{\theta}}+\mathrm{\dot{\phi}}\sin\left(\beta \right)\right)}^2+{\mathrm{\dot{\phi}}}^2{\cos\left(\beta \right)}^2{\sin\left(\phi \right)}^2\right)\right]}_{\text{B}}
         dx_2dx_1
    \end{split}
\end{equation}

Starting with the inner integral of A:

\begin{equation}
    \begin{split}
        &-c\int_{-L}^0
        \left[\frac{1}{3\dot\theta\cos(\beta)}{\left(L\mathrm{\dot{\theta}}+\mathrm{\dot{\theta}}x_{2}\cos\left(\beta \right)-\mathrm{\dot{\phi}}x_{1}\cos\left(\phi \right)+L\mathrm{\dot{\theta}}\cos\left(\beta \right)-\mathrm{\dot{\theta}}x_{1}\cos\left(\phi \right)\sin\left(\beta \right)\right)}^3\right]
        _{-\frac{L}{2}}^\frac{L}{2}dx_1 = \\
        &-c\int_{-L}^0
        \left[\frac{1}{3\dot\theta\cos(\beta)}{\left(L\mathrm{\dot{\theta}}+\mathrm{\dot{\theta}}x_{2}\cos\left(\beta \right)-\mathrm{\dot{\phi}}x_{1}\cos\left(\phi \right)+L\mathrm{\dot{\theta}}\cos\left(\beta \right)-\mathrm{\dot{\theta}}x_{1}\cos\left(\phi \right)\sin\left(\beta \right)\right)}^3\right]
        dx_1
    \end{split}
\end{equation}





    


