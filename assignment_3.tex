A few notes before starting this assignment:

I'm considering four frames. 

The first one is the inertial frame. It is in A with the first axis straight up, the second right and the third into the plane of the image.

\begin{figure}[ht]
    \centering
    \includegraphics[scale=0.5]{images/inertial_frame.png}
    \caption{Inertial Frame}
    %label always in the end
    \label{fig:inertial_frame}
\end{figure}

\noindent The second one is located at the same position but rotating with Omega. I will call it the vertical frame.

\begin{figure}[ht]
    \centering
    \includegraphics[scale=0.5]{images/vertical_frame.png}
    \caption{Vertical Frame}
    %label always in the end
    \label{fig:vertical_frame}
\end{figure}
\clearpage %HERE

The third one, that is called rod frame is located in the point B and relative to the vertical frame it is tilted by $\beta$ around the positive $e_3^v$ axis.

\begin{figure}[ht]
    \centering
    \includegraphics[scale=0.5]{images/rod_frame.png}
    \caption{Rod Frame}
    %label always in the end
    \label{fig:rod_frame}
\end{figure}

The last frame is the square frame which relative to the rod frame is also rotation with $\phi$:

\begin{figure}[ht]
    \centering
    \includegraphics[scale=0.5]{images/square_frame.png}
    \caption{Square Frame}
    %label always in the end
    \label{fig:square_frame}
\end{figure}


\subsection{Kinetic and Potential Energy}
\subsubsection{Vertical Shaft}
As the vertical bar rotates around the thin axis and is static regarding the height of it's CoM the contribution to the energy is 0.

\begin{equation}
    \begin{split}
        T_\text{vertical bar} =  V_\text{vertical bar} = 0
    \end{split}
\end{equation}
\subsubsection{Horizontal Bar}
\textbf{Note: } I will refer to the horizontal bar as bar and to the tilted bar as rod.

As point A which is part of the bar is not moving we can simply consider the rotational part w.r.t. to this fixed point A.:

\begin{equation}
    \begin{split}
        T_\text{bar} =  \frac{L^2\Omega ^2m}{6}
    \end{split}
\end{equation}

As the bar is horizontal on the height of A we have no contribution of the potential energy.

\begin{equation}
    \begin{split}
        V_\text{bar} = 0
    \end{split}
\end{equation}

\subsubsection{Rod}
As the rod has no static point we will use a superposition of the rotational and translational kinetic energy.

\begin{equation}
    \begin{split}
        T_\text{Rod transl} = \frac{L^2\Omega ^2m{\left(\cos\left(\beta \right)+2\right)}^2}{8}
    \end{split}
\end{equation}
For the rotational part considering the moment of inertia of the center of mass we get:
\begin{equation}
    \begin{split}
        T_\text{Rod rot} = \frac{L^2\Omega ^2m{\cos\left(\beta \right)}^2}{24}
    \end{split}
\end{equation}
Which leads to:
\begin{equation}
    \begin{split}
        \Rightarrow T_\text{Rod} = \frac{L^2\Omega ^2m\left({\cos\left(\beta \right)}^2+3\cos\left(\beta \right)+3\right)}{6}
    \end{split}
\end{equation}

For the potential energy we have the verical part of the incline of the rod:

\begin{equation}
    \begin{split}
        V_\text{Rod} = \frac{L}{2}mg\sin\beta
    \end{split}
\end{equation}

\subsubsection{Square Plate}
As for the rod, first the translational kinetic energy:

\begin{equation}
    \begin{split}
        T_\text{square transl} = &\frac{M}{2}\left(\left(\vphantom{\int}\left(\Omega +\mathrm{\dot\phi}\sin\left(\beta \right)\right)\left(\frac{L\left(\sin\left(\Omega t\right)\sin\left(\phi \right)-\cos\left(\Omega t\right)\cos\left(\phi \right)\sin\left(\beta \right)\right)}{2}-\frac{L\cos\left(\Omega t\right)\cos\left(\beta \right)}{2}\right)\right.\right.\\
        &\left.-L\Omega \cos\left(\Omega t\right)+\frac{L\mathrm{\dot\phi}\cos\left(\Omega t\right)\cos\left(\beta \right)\left(\sin\left(\beta \right)-\cos\left(\beta \right)\cos\left(\phi \right)\right)}{2}\right)^2\\
        &+\left(\left(\Omega +\mathrm{\dot\phi}\sin\left(\beta \right)\right)\left(\frac{L\left(\cos\left(\Omega t\right)\sin\left(\phi \right)+\sin\left(\Omega t\right)\cos\left(\phi \right)\sin\left(\beta \right)\right)}{2}+\frac{L\sin\left(\Omega t\right)\cos\left(\beta \right)}{2}\right)\right.\\
        &\left.\left.+L\Omega \sin\left(\Omega t\right)-\frac{L\mathrm{\dot\phi}\sin\left(\Omega t\right)\cos\left(\beta \right)\left(\sin\left(\beta \right)-\cos\left(\beta \right)\cos\left(\phi \right)\right)}{2}\right)^2+\frac{L^2{\mathrm{\dot\phi}}^2{\cos\left(\beta \right)}^2{\sin\left(\phi \right)}^2}{4}\right)
    \end{split}
\end{equation}

For the rotational part we get, considering the moment of inertia w.r.t. the c.o.m. in the s frame:

\begin{equation}
    \begin{split}
        T_\text{square rot} = &\frac{L^2M\left(\mathrm{\dot\phi}+\Omega \sin\left(\beta \right)\right)\left(\frac{\mathrm{\dot\phi}}{2}+\frac{\Omega \sin\left(\beta \right)}{2}\right)}{12}\\
        &+\frac{L^2M\Omega ^2{\cos\left(\beta \right)}^2{\cos\left(\phi \right)}^2}{24}
    \end{split}
\end{equation}

Which leads to:

\begin{equation}
    \begin{split}
        T_\text{square} = &\frac{L^2M}{12}\left(-\Omega ^2{\cos\left(\beta \right)}^2{\cos\left(\phi \right)}^2+\Omega ^2{\cos\left(\beta \right)}^2\right.\\
        &+3\sin\left(\beta \right)\Omega ^2\cos\left(\beta \right)\cos\left(\phi \right)+6\Omega ^2\cos\left(\beta \right)\\
        &+6\sin\left(\beta \right)\Omega ^2\cos\left(\phi \right)+8\Omega ^2+3\Omega \mathrm{\dot\phi}\cos\left(\beta \right)\cos\left(\phi \right)\\
        &\left.\vphantom{\int}+6\Omega \mathrm{\dot\phi}\cos\left(\phi \right)+4\sin\left(\beta \right)\Omega \mathrm{\dot\phi}+2{\mathrm{\dot\phi}}^2\right)
    \end{split}
\end{equation}

For the potential energy we get:
\begin{equation}
    \begin{split}
        \frac{L}{2}Mg\left(\sin\left(\beta \right)-\cos\left(\beta \right)\cos\left(\phi \right)\right)
    \end{split}
\end{equation}

Which can be seen as the contribution of the axis-parallel parts of the vector to the center of mass of the square in the s frame. The sin part points up (to c.o.m of the rod) and the cos part points down.

\subsubsection{Total Energy}
The total kinetic energy comes to:

\begin{equation}
    \begin{split}
        &T = \underbrace{T_\text{vertical bar}}_{0} + T_\text{bar} + T_\text{Rod} + T_\text{Square} = \\
        &\frac{L^2}{12}\left(8M\Omega ^2+8\Omega ^2m+2M{\mathrm{\dot\phi}}^2+6M\Omega ^2\cos\left(\beta \right)+6\Omega ^2m\cos\left(\beta \right)\right.\\
        &+M\Omega ^2{\cos\left(\beta \right)}^2+2\Omega ^2m{\cos\left(\beta \right)}^2+6M\Omega ^2\cos\left(\phi \right)\sin\left(\beta \right)\\
        &+6M\Omega \mathrm{\dot\phi}\cos\left(\phi \right)+4M\Omega \mathrm{\dot\phi}\sin\left(\beta \right)-M\Omega ^2{\cos\left(\beta \right)}^2{\cos\left(\phi \right)}^2\\
        &\left.\vphantom{\int}+3M\Omega \mathrm{\dot\phi}\cos\left(\beta \right)\cos\left(\phi \right)+3M\Omega ^2\cos\left(\beta \right)\cos\left(\phi \right)\sin\left(\beta \right)\right)
    \end{split}
\end{equation}

Analogous the potential energy:
\begin{equation}
    \begin{split}
        &V = \underbrace{V_\text{vertical bar}}_{0} + V_\text{bar} + V_\text{Rod} + V_\text{Square} = \\
        & \frac{L}{2}Mg\left(2\sin\left(\beta \right)-\cos\left(\beta \right)\cos\left(\phi \right)\right)
    \end{split}
\end{equation}

Where one can see nicely the double contribution of the c.o.m of the rod for the rod and the square and the negative part of the square.


\subsection{Lagrange Equations}
Having an expression for the kinetic and potential energy we can use them to denote our lagrange equations:

\begin{equation}
    \begin{split}
        \frac{\partial}{\partial t}\frac{\partial T}{\partial \dot\phi} - \frac{\partial T}{\partial \phi} + \frac{\partial V}{\partial \phi} = 0
    \end{split}
\end{equation}

Which yields:

\begin{equation}
    \begin{split}
        &\frac{LM}{12}\left(-2L\cos\left(\phi \right)\sin\left(\phi \right)\Omega ^2{\cos\left(\beta \right)}^2+3L\sin\left(\beta \right)\sin\left(\phi \right)\Omega ^2\cos\left(\beta \right)\right.\\
        &\left.\vphantom{\int}+6L\sin\left(\beta \right)\sin\left(\phi \right)\Omega ^2+6g\sin\left(\phi \right)\cos\left(\beta \right)+4L\mathrm{\ddot\phi}\right) = 0
    \end{split}
\end{equation}

Which is a differential equation for $\phi$.

\subsection{Non constant angular velocity around vertical Shaft}
\subsubsection{Vertical shaft}
Unchanged
\subsubsection{Bar}

 \begin{equation}
    \begin{split}
        T_\text{bar} = \frac{L^2\Omega ^2m}{6}
    \end{split}
 \end{equation}

 \begin{equation}
    \begin{split}
        V_\text{bar} = 0
    \end{split}
 \end{equation}

 \subsubsection{Rod}

 \begin{equation}
    \begin{split}
        T_\text{rod} = L^2m\mathrm{\dot\theta}^2\left(\frac{\cos\left(\beta \right)^2}{2}+\cos\left(\beta \right)+1\right)
    \end{split}
 \end{equation}

 \begin{equation}
    \begin{split}
        V_\text{rod} = \frac{L}{2}gm\sin\left(\beta \right)
    \end{split}
 \end{equation}

 \subsubsection{Square}

 \begin{equation}
    \begin{split}
        T_\text{square} = &\frac{L^2M}{12}\left(2{\mathrm{\dot\phi}}^2+3\mathrm{\dot\phi}\mathrm{\dot\theta}\cos\left(\beta \right)\cos\left(\phi \right)+6\mathrm{\dot\phi}\mathrm{\dot\theta}\cos\left(\phi \right)+4\sin\left(\beta \right)\mathrm{\dot\phi}\mathrm{\dot\theta}\right.\\
        &-{\mathrm{\dot\theta}}^2{\cos\left(\beta \right)}^2{\cos\left(\phi \right)}^2+{\mathrm{\dot\theta}}^2{\cos\left(\beta \right)}^2+3\sin\left(\beta \right){\mathrm{\dot\theta}}^2\cos\left(\beta \right)\cos\left(\phi \right)\\
        &\left.\vphantom{\int}+6{\mathrm{\dot\theta}}^2\cos\left(\beta \right)+6\sin\left(\beta \right){\mathrm{\dot\theta}}^2\cos\left(\phi \right)+8{\mathrm{\dot\theta}}^2\right)
    \end{split}
 \end{equation}

 \begin{equation}
    \begin{split}
        V_\text{square} = \frac{L}{2}Mg\left(\sin\left(\beta \right)-\cos\left(\beta \right)\cos\left(\phi \right)\right)
    \end{split}
 \end{equation}

 \subsubsection{Total Energy}

 \begin{equation}
    \begin{split}
        T = &\frac{L^2}{12}\left(2M{\mathrm{\dot\phi}}^2+8M{\mathrm{\dot\theta}}^2+8m{\mathrm{\dot\theta}}^2+6M{\mathrm{\dot\theta}}^2\cos\left(\beta \right)+6m{\mathrm{\dot\theta}}^2\cos\left(\beta \right)\right.\\
        &+M{\mathrm{\dot\theta}}^2{\cos\left(\beta \right)}^2+2m{\mathrm{\dot\theta}}^2{\cos\left(\beta \right)}^2+6M{\mathrm{\dot\theta}}^2\cos\left(\phi \right)\sin\left(\beta \right)\\
        &+6M\mathrm{\dot\phi}\mathrm{\dot\theta}\cos\left(\phi \right)+4M\mathrm{\dot\phi}\mathrm{\dot\theta}\sin\left(\beta \right)-M{\mathrm{\dot\theta}}^2{\cos\left(\beta \right)}^2{\cos\left(\phi \right)}^2\\
        &\left.\vphantom{\int}+3M\mathrm{\dot\phi}\mathrm{\dot\theta}\cos\left(\beta \right)\cos\left(\phi \right)+3M{\mathrm{\dot\theta}}^2\cos\left(\beta \right)\cos\left(\phi \right)\sin\left(\beta \right)\right)
    \end{split}
 \end{equation}

 \begin{equation}
    \begin{split}
        V = \frac{L}{2}gm\sin\left(\beta \right)+\frac{L}{2}Mg\left(\sin\left(\beta \right)-\cos\left(\beta \right)\cos\left(\phi \right)\right)
    \end{split}
 \end{equation}

 \subsubsection{Lagrange Equations}

%  As now we have two generalized coordinates we also have two lagrangr equations. For $\phi$ we get:

%  \begin{equation}
%     \begin{split}
%         &\frac{LM}{12}\left(-2L\cos\left(\phi \right)\sin\left(\phi \right){\mathrm{\dot\theta}}^2{\cos\left(\beta \right)}^2+3L\sin\left(\beta \right)\sin\left(\phi \right){\mathrm{\dot\theta}}^2\cos\left(\beta \right)\right.\\
%         &\left.\vphantom{\int}+6L\sin\left(\beta \right)\sin\left(\phi \right){\mathrm{\dot\theta}}^2+6g\sin\left(\phi \right)\cos\left(\beta \right)+4L\mathrm{\ddot\phi}\right)
%     \end{split}
%  \end{equation}

%  And for $\theta$:

%  \begin{equation}
%     \begin{split}
%         &\frac{L^2}{12}\mathrm{\ddot\theta}\left(16M+16m+12M\cos\left(\beta \right)+12m\cos\left(\beta \right)+2M{\cos\left(\beta \right)}^2\right.\\
%         &+4m{\cos\left(\beta \right)}^2-2M{\cos\left(\beta \right)}^2{\cos\left(\phi \right)}^2+12M\cos\left(\phi \right)\sin\left(\beta \right)\\
%         &\left.\vphantom{\int}+6M\cos\left(\beta \right)\cos\left(\phi \right)\sin\left(\beta \right)\right)
%     \end{split}
%  \end{equation}

As we have now two generalized coordinates the lagrange equations look like this:

\begin{equation}
    \begin{split}
        \frac{\partial}{\partial t}\frac{\partial T}{\partial  \text{\textbf{q}}} - \frac{\partial T}{\partial \text{\textbf{q}}} + \frac{\partial V}{\partial \text{\textbf{q}}} = 0
    \end{split}
\end{equation}

Where $\text{\textbf{q}} = \begin{pmatrix}
    \phi\\
    \theta
\end{pmatrix}$

This leads to two equations. For visualization purposes I will write down the entries one after the other:

\begin{equation}
    \begin{split}
        &\frac{L^2\,M\,\mathrm{\ddot \phi}}{3}+\frac{L^2\,M\,\mathrm{\ddot  \theta}\,\left(6\,\cos\left(\phi \right)+4\,\sin\left(\beta \right)+3\,\cos\left(\beta \right)\,\cos\left(\phi \right)\right)}{12}+\frac{L\,M\,g\,\cos\left(\beta \right)\,\sin\left(\phi \right)}{2}\\
        &+\frac{L^2\,M\,\mathrm{\dot  \theta}\,\sin\left(\phi \right)\,\left(6\,\mathrm{\dot \phi}+3\,\mathrm{\dot \phi}\,\cos\left(\beta \right)+6\,\mathrm{\dot  \theta}\,\sin\left(\beta \right)+3\,\mathrm{\dot  \theta}\,\cos\left(\beta \right)\,\sin\left(\beta \right)-2\,\mathrm{\dot  \theta}\,{\cos\left(\beta \right)}^2\,\cos\left(\phi \right)\right)}{12}\\
        &-\frac{L^2\,M\,\mathrm{\dot \phi}\,\mathrm{\dot  \theta}\,\sin\left(\phi \right)\,\left(\cos\left(\beta \right)+2\right)}{4}
    \end{split}
\end{equation}

And

\begin{equation}
    \begin{split}
        &\frac{L^2\,\mathrm{\ddot  \theta}}{12}\,\left(\vphantom{\int}16\,M+16\,m+12\,M\,\cos\left(\beta \right)+12\,m\,\cos\left(\beta \right)+2\,M\,{\cos\left(\beta \right)}^2\right.\\
        &\left.\vphantom{\int}+4\,m\,{\cos\left(\beta \right)}^2-2\,M\,{\cos\left(\beta \right)}^2\,{\cos\left(\phi \right)}^2+12\,M\,\cos\left(\phi \right)\,\sin\left(\beta \right)+6\,M\,\cos\left(\beta \right)\,\cos\left(\phi \right)\,\sin\left(\beta \right)\right)\\
        &+\frac{L^2\,M\,\mathrm{\ddot \phi}\,\left(6\,\cos\left(\phi \right)+4\,\sin\left(\beta \right)+3\,\cos\left(\beta \right)\,\cos\left(\phi \right)\right)}{12}\\
        &-\frac{L^2\,M\,\mathrm{\dot \phi}\,\sin\left(\phi \right)\,\left(6\,\mathrm{\dot \phi}+3\,\mathrm{\dot \phi}\,\cos\left(\beta \right)+12\,\mathrm{\dot  \theta}\,\sin\left(\beta \right)+6\,\mathrm{\dot  \theta}\,\cos\left(\beta \right)\,\sin\left(\beta \right)-4\,\mathrm{\dot  \theta}\,{\cos\left(\beta \right)}^2\,\cos\left(\phi \right)\right)}{12}
    \end{split}
\end{equation}



\subsection{Non Conservative Forces}\label{subsec:3.4}
Now we have a non conservative force attacking on the square.

We have to derive the velocity of a random point on the square. In the following I will parametrize said point with $x_1$ and $x_2$ which are the coordinates of the point on the square in the s frame from B.

\begin{equation}
    \begin{split}
        v_{Pi} &= v_{Bi} + \omega_i \times BP_i
        % \\&=\begin{pmatrix}
        %     &-\mathrm{\dot\phi}x_{1}\cos\left(\beta \right)\sin\left(\phi \right)\\
        %     &\left(x_{1}\left(\cos\left(\theta \right)\sin\left(\phi \right)+\cos\left(\phi \right)\sin\left(\beta \right)\sin\left(\theta \right)\right)-x_{2}\cos\left(\beta \right)\sin\left(\theta \right)\right)\left(\mathrm{\dot\theta}+\mathrm{\dot\phi}\sin\left(\beta \right)\right)-L\mathrm{\dot\theta}\sin\left(\theta \right)+\mathrm{\dot\phi}\cos\left(\beta \right)\sin\left(\theta \right)\left(x_{2}\sin\left(\beta \right)+x_{1}\cos\left(\beta \right)\cos\left(\phi \right)\right)\\
        %     &\left(x_{1}\left(\sin\left(\phi \right)\sin\left(\theta \right)-\cos\left(\phi \right)\sin\left(\beta \right)\cos\left(\theta \right)\right)+x_{2}\cos\left(\beta \right)\cos\left(\theta \right)\right)\left(\mathrm{\dot\theta}+\mathrm{\dot\phi}\sin\left(\beta \right)\right)+L\mathrm{\dot\theta}\cos\left(\theta \right)-\mathrm{\dot\phi}\cos\left(\beta \right)\cos\left(\theta \right)\left(x_{2}\sin\left(\beta \right)+x_{1}\cos\left(\beta \right)\cos\left(\phi \right)\right)
        % \end{pmatrix}
    \end{split}
\end{equation}

Therefore the work of the force contribution at a single point P is:

\begin{equation}
    \begin{split}
        dW = &-cvP_i\bullet vP_i = \\
        &= -c\left(\vphantom{\int}\left(\vphantom{\sum}x_{1}\left(\sin\left(\phi \right)\sin\left(\theta \right)-\cos\left(\phi \right)\sin\left(\beta \right)\cos\left(\theta \right)\right)\right.\right.\\
        &\left.\vphantom{\sum}+x_{2}\cos\left(\beta \right)\cos\left(\theta \right)\right)\left(\mathrm{\dot\theta}+\mathrm{\dot\phi}\sin\left(\beta \right)\right)+L\mathrm{\dot\theta}\cos\left(\theta \right)\\
        &\left.\vphantom{\int}-\mathrm{\dot\phi}\cos\left(\beta \right)\cos\left(\theta \right)\left(x_{2}\sin\left(\beta \right)+x_{1}\cos\left(\beta \right)\cos\left(\phi \right)\right)\right)^2\\
        &-c\left(\vphantom{\int}\left(\vphantom{\sum}x_{1}\left(\cos\left(\theta \right)\sin\left(\phi \right)+\cos\left(\phi \right)\sin\left(\beta \right)\sin\left(\theta \right)\right)-x_{2}\cos\left(\beta \right)\sin\left(\theta \right)\right)\right.\\
&\left(\mathrm{\dot\theta}+\mathrm{\dot\phi}\sin\left(\beta \right)\right)-L\mathrm{\dot\theta}\sin\left(\theta \right)\\
&\left.\vphantom{\sum}+\mathrm{\dot\phi}\cos\left(\beta \right)\sin\left(\theta \right)\left(x_{2}\sin\left(\beta \right)+x_{1}\cos\left(\beta \right)\cos\left(\phi \right)\vphantom{\sum}\right)\vphantom{\int}\right)^2\\
&-c{\mathrm{\dot\phi}}^2{x_{1}}^2{\cos\left(\beta \right)}^2\sin\left(\phi \right)^2
    \end{split}
\end{equation}

As we want to derive the whole work done by the area distributed force we integrate the expression of dW over the whole square:

\begin{equation}
    \begin{split}
        W = &\int_{x_1 = -L}^0\int_{x_2 = 0}^L-c\left(\vphantom{\int}\left(\vphantom{\sum}x_{1}\left(\sin\left(\phi \right)\sin\left(\theta \right)-\cos\left(\phi \right)\sin\left(\beta \right)\cos\left(\theta \right)\right)\right.\right.\\
        &\left.\vphantom{\sum}+x_{2}\cos\left(\beta \right)\cos\left(\theta \right)\right)\left(\mathrm{\dot\theta}+\mathrm{\dot\phi}\sin\left(\beta \right)\right)+L\mathrm{\dot\theta}\cos\left(\theta \right)\\
        &\left.\vphantom{\int}-\mathrm{\dot\phi}\cos\left(\beta \right)\cos\left(\theta \right)\left(x_{2}\sin\left(\beta \right)+x_{1}\cos\left(\beta \right)\cos\left(\phi \right)\right)\right)^2\\
        &-c\left(\vphantom{\int}\left(\vphantom{\sum}x_{1}\left(\cos\left(\theta \right)\sin\left(\phi \right)+\cos\left(\phi \right)\sin\left(\beta \right)\sin\left(\theta \right)\right)-x_{2}\cos\left(\beta \right)\sin\left(\theta \right)\right)\right.\\
&\left(\mathrm{\dot\theta}+\mathrm{\dot\phi}\sin\left(\beta \right)\right)-L\mathrm{\dot\theta}\sin\left(\theta \right)\\
&\left.\vphantom{\sum}+\mathrm{\dot\phi}\cos\left(\beta \right)\sin\left(\theta \right)\left(x_{2}\sin\left(\beta \right)+x_{1}\cos\left(\beta \right)\cos\left(\phi \right)\vphantom{\sum}\right)\vphantom{\int}\right)^2\\
&-c{\mathrm{\dot\phi}}^2{x_{1}}^2{\cos\left(\beta \right)}^2\sin\left(\phi \right)^2dx_2dx_1
    \end{split}
\end{equation}

Which comes to:

\begin{equation}
    \begin{split}
        W = &-\frac{cL^4{\mathrm{\dot\phi}}^2}{3}-\frac{cL^4\mathrm{\dot\phi}\mathrm{\dot\theta}\cos\left(\beta \right)\cos\left(\phi \right)}{2}-cL^4\mathrm{\dot\phi}\mathrm{\dot\theta}\cos\left(\phi \right)-\\
        &\frac{2c\sin\left(\beta \right)L^4\mathrm{\dot\phi}\mathrm{\dot\theta}}{3}+\frac{cL^4{\mathrm{\dot\theta}}^2{\cos\left(\beta \right)}^2{\cos\left(\phi \right)}^2}{3}-\frac{cL^4{\mathrm{\dot\theta}}^2{\cos\left(\beta \right)}^2}{3}-\\
        &\frac{c\sin\left(\beta \right)L^4{\mathrm{\dot\theta}}^2\cos\left(\beta \right)\cos\left(\phi \right)}{2}-cL^4{\mathrm{\dot\theta}}^2\cos\left(\beta \right)-c\sin\left(\beta \right)L^4{\mathrm{\dot\theta}}^2\cos\left(\phi \right)\\
        &-\frac{4cL^4{\mathrm{\dot\theta}}^2}{3}
    \end{split}
\end{equation}

And finally the generalized forces are:

\begin{equation}
    \begin{split}
        Q_\phi = \frac{\partial W}{\partial \phi} = -\frac{L^4c\left(4\mathrm{\dot\phi}+6\mathrm{\dot\theta}\cos\left(\phi \right)+4\mathrm{\dot\theta}\sin\left(\beta \right)+3\mathrm{\dot\theta}\cos\left(\beta \right)\cos\left(\phi \right)\right)}{6}
    \end{split}
\end{equation}

And 

\begin{equation}
    \begin{split}
        Q_\theta = &-\frac{L^4c}{6}\left(\vphantom{\int}16\mathrm{\dot\theta}+12\mathrm{\dot\theta}\cos\left(\beta \right)+6\mathrm{\dot\phi}\cos\left(\phi \right)+4\mathrm{\dot\phi}\sin\left(\beta \right)+4\mathrm{\dot\theta}{\cos\left(\beta \right)}^2\right.\\
        &-4\mathrm{\dot\theta}{\cos\left(\beta \right)}^2{\cos\left(\phi \right)}^2+3\mathrm{\dot\phi}\cos\left(\beta \right)\cos\left(\phi \right)+12\mathrm{\dot\theta}\cos\left(\phi \right)\sin\left(\beta \right)\\
        &\left.\vphantom{\int}+6\mathrm{\dot\theta}\cos\left(\beta \right)\cos\left(\phi \right)\sin\left(\beta \right)\right)
    \end{split}
\end{equation}

\subsection{Values plugged in}
We plug in the values:
\begin{equation}
    \begin{split}
        L=0.25,\beta = \pi/6, M= 0.5, m = 0.2, g = 9.81, c = 0.1
    \end{split}
\end{equation}
\subsection{Equilibrium}
In the equilibrium configuration we have

\begin{equation}
    \begin{split}
        \ddot \phi = \dot \phi = 0, \quad \phi = \phi_\text{eq}
    \end{split}
\end{equation}

And

\begin{equation}
    \begin{split}
        \ddot \theta = \dot \theta = 0, \quad \theta = \theta_\text{eq}
    \end{split}
\end{equation}

For the rheonomic system with a constant Omega and the plugged in values we get:

\begin{equation}\label{eq:3.6.3}
    \begin{split}
        \frac{981\sqrt{3}\sin\left(\phi \right)}{3200}+\frac{\Omega ^2\sin\left(\phi \right)}{128}+\frac{\sqrt{3}\Omega ^2\sin\left(\phi \right)}{512}-\frac{\Omega ^2\cos\left(\phi \right)\sin\left(\phi \right)}{256} = 0
    \end{split}
\end{equation}

Or for the case of a free rotation around A:

\begin{equation}\label{eq:3.6.4}
    \begin{split}
        \left(\begin{array}{c} \frac{981\,\sqrt{3}\,\sin\left(\phi _{\mathrm{eq}}\right)}{3200}\\ 0 \end{array}\right) = \left(\begin{array}{c}0\\ 0 \end{array}\right)
    \end{split}
\end{equation}

And as can be seen the above equation (\ref{eq:3.6.4}) can be read as:

\begin{equation}\label{eq:3.6.5}
    \begin{split}
        \begin{pmatrix}
            A\sin(\phi_{\text{eq}})\\0
        \end{pmatrix} = \begin{pmatrix}
            0\\0
        \end{pmatrix}
    \end{split}
\end{equation}

With a constant A. Which yields 
\begin{equation}
    \begin{split}
        \phi_{\text{eq}} = n*\pi, \quad \text{ for } n = 0,1,2...
    \end{split}
\end{equation}

% \subsubsection{$\phi$ - Equilibrium}

% From equation (\ref{eq:3.6.3}) we can see that the term $\sin\phi$ is present in every part of the expression so that we can say the equilibrium holds for $\phi = n*\pi$, for $n = 0,1,2,...$

This makes sense as $\phi = n*\pi$ describes states where the square plate stays vertical. As we are not considering aerial drag (non-conservative forces) it is imaginable that this is an equilibrium. $\theta$ can be chosen arbitrarily which makes sense as the system is symmetric w.r.t. theta.

\textbf{Note:} as for this case we have a scleronomic system we could have also just used the following:
\begin{equation}
    \begin{split}
        \frac{\partial V}{\partial \text{\textbf{q}}} = \text{\textbf{0}}
    \end{split}
\end{equation}

Which when plugging it in gives us:
\begin{equation}
    \left(\begin{array}{cc} \frac{981\,\sqrt{3}\,\sin\left(\phi _{\mathrm{eq}}\right)}{3200} & 0 \end{array}\right) = \text{\textbf{0}}
\end{equation}

Which is the same as equation (\ref{eq:3.6.4})
% The same holds for a free rotation around A as can bee seen in equation (\ref{eq:3.6.4}).
% \subsubsection{$\theta$ - Equilibrium}

% Solving equation (\ref{eq:3.6.5}) we can see that $\ddot \theta$ has to be 0 and we thus recover a constant angular velocity around A. This makes sense as a constant, unconstrained velocity doesn't generate any forces. Note that if we considered the non-conservative forces from step \ref{subsec:3.4} that the equilibrium would be different.

% \begin{equation}
%     \begin{split}
%         \begin{pmatrix} &\frac{L^2M\mathrm{\ddot \phi}}{3}+\frac{L^2M\mathrm{\ddot \theta}\left(6\cos\left(\phi \right)+4\sin\left(\beta \right)+3\cos\left(\beta \right)\cos\left(\phi \right)\right)}{12}+\frac{LMg\cos\left(\beta \right)\sin\left(\phi \right)}{2}+\frac{L^2M\mathrm{\dot \theta}\sin\left(\phi \right)\left(6\mathrm{\dot \phi}+3\mathrm{\dot \phi}\cos\left(\beta \right)+6\mathrm{\dot \theta}\sin\left(\beta \right)+3\mathrm{\dot \theta}\cos\left(\beta \right)\sin\left(\beta \right)-2\mathrm{\dot \theta}{\cos\left(\beta \right)}^2\cos\left(\phi \right)\right)}{12}-\frac{L^2M\mathrm{\dot \phi}\mathrm{\dot \theta}\sin\left(\phi \right)\left(\cos\left(\beta \right)+2\right)}{4}\\ 
%             &\frac{L^2\mathrm{\ddot \theta}\left(16M+16m+12M\cos\left(\beta \right)+12m\cos\left(\beta \right)+2M{\cos\left(\beta \right)}^2+4m{\cos\left(\beta \right)}^2-2M{\cos\left(\beta \right)}^2{\cos\left(\phi \right)}^2+12M\cos\left(\phi \right)\sin\left(\beta \right)+6M\cos\left(\beta \right)\cos\left(\phi \right)\sin\left(\beta \right)\right)}{12}+\frac{L^2M\mathrm{\ddot \phi}\left(6\cos\left(\phi \right)+4\sin\left(\beta \right)+3\cos\left(\beta \right)\cos\left(\phi \right)\right)}{12}-\frac{L^2M\mathrm{\dot \phi}\sin\left(\phi \right)\left(6\mathrm{\dot \phi}+3\mathrm{\dot \phi}\cos\left(\beta \right)+12\mathrm{\dot \theta}\sin\left(\beta \right)+6\mathrm{\dot \theta}\cos\left(\beta \right)\sin\left(\beta \right)-4\mathrm{\dot \theta}{\cos\left(\beta \right)}^2\cos\left(\phi \right)\right)}{12} \end{pmatrix}
%     \end{split}
% \end{equation}

Lastly when considering the non-conservative forces as well we get:

\begin{equation}
    \begin{split}
        \left(\begin{array}{c} \frac{981\,\sqrt{3}\,\sin\left(\phi _{\mathrm{eq}}\right)}{3200}\\ 0 \end{array}\right)
    \end{split}
\end{equation}

Which is the same equilibrium configuration as for the case with the free rotation of the vertical shaft.
\clearpage
%\HERE
Looking at the equations of motion for the case of constant rotation around the vertical shaft whilst considering the non-conservative forces:

\begin{equation}
    \begin{split}
        &\frac{\Omega }{7680}+\frac{\Omega \,\cos\left(\phi _{\mathrm{eq}}\right)}{2560}+\frac{981\,\sqrt{3}\,\sin\left(\phi _{\mathrm{eq}}\right)}{3200}+\frac{\Omega ^2\,\sin\left(\phi _{\mathrm{eq}}\right)}{128}+\frac{\sqrt{3}\,\Omega ^2\,\sin\left(\phi _{\mathrm{eq}}\right)}{512}\\
        &-\frac{\Omega ^2\,\cos\left(\phi _{\mathrm{eq}}\right)\,\sin\left(\phi _{\mathrm{eq}}\right)}{256}+\frac{\sqrt{3}\,\Omega \,\cos\left(\phi _{\mathrm{eq}}\right)}{10240} = 0
    \end{split}
\end{equation}

From the first view we can see that $\phi = 0$ is not the solution anymore which makes sense because the non-conservative forces simulate aerial drag which would induce a tilt in the square when rotating at a constant angular velocity.

Trying to solve this equation yielded no solution for $\phi_{eq}$. This can be interpreted as follows: We have a constant angular velocity around the vertical shaft. This rotation induces a velocity in the bar, the rod and the square plate. This generates a drag force on the square plate which will remain there as long as there is an angular velocity and therefore no equilibrium can be reached.



\subsection{Integration}

To perform the time integration of the state space representation we use the same approach as given in the car exercise. Aka we say that:
\begin{equation}
    \begin{split}
        \begin{bmatrix}
            M(q) & 0 \\
            0 I
        \end{bmatrix}\begin{bmatrix}
            \ddot q\\\dot q
        \end{bmatrix} = \begin{bmatrix}
            f\\\dot q
        \end{bmatrix}
    \end{split}
\end{equation}
Where $f = -(\text{equations of motion} - M(q)\ddot q)$

The result of this time integration for iniitial conditions of:
\begin{equation}
    \begin{split}
        \begin{pmatrix}
            \dot \phi\\
            \dot \theta\\
            \phi\\
            \theta
        \end{pmatrix} = \begin{pmatrix}
            \pi /s\\
            0\\
            0\\
            0
        \end{pmatrix}
    \end{split}
\end{equation}
\clearpage %HERE

is the following for phi:
\begin{figure}[ht]
    \centering
    \includegraphics[scale=0.35]{images/phid_case_1.jpg}
    \caption{$\dot\phi$ for initial rotation of the square plate}
    \label{fig:phid_case1}
\end{figure}
\clearpage %HERE
and this for theta:
\begin{figure}[ht]
    \centering
    \includegraphics[scale=0.35]{images/thetad_case_1.jpg}
    \caption{$\dot\theta$ for initial rotation of the square plate}
    \label{fig:thetad_case1}
\end{figure}

\textbf{Interpretation:}

With initial conditions of only a phi rotation we get an induced angular momentum in the system which has to be compensated by the angular momentum around the vertical shaft. Therefore we get a precession around the vertical shaft. The induced precession in turn leads to a rotation of the square plate in the opposite direction and so on. This highly oscillatory movement can be seen in figures \ref{fig:phid_case1} and \ref{fig:thetad_case1}. Also it can be noted that the amplitude of the oscillation diminishes for both rotations. This is due to the drag forces on the square plate.
\clearpage %HERE
And for the second initial conditions of exciting the theta around the vertical shaft:


\begin{equation}
    \begin{split}
        \begin{pmatrix}
            \dot \phi\\
            \dot \theta\\
            \phi\\
            \theta
        \end{pmatrix} = \begin{pmatrix}
            0\\
            \pi /s\\
            0\\
            0
        \end{pmatrix}
    \end{split}
\end{equation}

we get for phi:
\begin{figure}[ht]
    \centering
    \includegraphics[scale=0.35]{images/phid_case_2.jpg}
    \caption{$\dot\phi$ for initial rotation of the vertical shaft}
    \label{fig:phid_case2}
\end{figure}
\clearpage %HERE
and for theta:
\begin{figure}[ht]
    \centering
    \includegraphics[scale=0.35]{images/thetad_case_2.jpg}
    \caption{$\dot\theta$ for initial rotation of the vertical shaft}
    \label{fig:thetad_case2}
\end{figure}

\textbf{Interpretation}

Here we have a smooth rotation of the system around the vertical shaft. The angular velocity of theta goes towards 0 as can be seen in figure \ref{fig:thetad_case2} this is of course again due to the drag force acting on the smoothly rotating system. The angular velocity of the square plate is again oscillatory but with very small magnitudes. This is the case because the square plate tries to find it's equilibrium position during the rotation around the vertical shaft but can't as the drag force pusheds the plate up and gravity pulls it down and $\dot\theta$ is diminishing so the equilibrium always changes aka there is no equilibrium. In the end this looks like a rotation around the vertical shaft with a tilted square plate that wobbles in the wind. 

To proof this we can also plot the angle of the square plate over time:
\clearpage %HERE
\begin{figure}[ht]
    \centering
    \includegraphics[scale=0.35]{images/phi_case_2.jpg}
    \caption{$\phi$ for initial rotation of the vertical shaft}
    \label{fig:phi_case_2}
\end{figure}
And as expected we have very tiny fluctuations (range 1e-4) of the $\phi$ angle.