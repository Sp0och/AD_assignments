\subsection{}
Lagrange equations:
\begin{equation}
    \begin{split}
        &\frac{L^2M\mathrm{\ddot{\phi}}}{3}+\frac{L^2m\mathrm{\ddot{\phi}}}{3}-\frac{L^2m\mathrm{\ddot{\phi}}{\cos\left(\beta \right)}^2}{3}+\frac{LMg\cos\left(\beta \right)\sin\left(\phi \right)}{2}+\frac{L^2M\Omega ^2\sin\left(\beta \right)\sin\left(\phi \right)}{2}+\\
        &\frac{L^2M\Omega ^2\cos\left(\beta \right)\sin\left(\beta \right)\sin\left(\phi \right)}{4}-\frac{L^2M\Omega ^2{\cos\left(\beta \right)}^2\cos\left(\phi \right)\sin\left(\phi \right)}{6} = 0
    \end{split}
\end{equation}
\subsection{}
This can be reformulated for the differential equation of $\ddot{\phi}$:

\begin{equation}
    \begin{split}
        &\ddot{\phi} \left(\frac{L^2(M+m)}{3} - \frac{L^2m\cos(\beta)^2}{3}\right) + \sin(\phi)\left(\frac{LMg\cos(\beta)}{2} + \frac{L^2M\Omega^2\sin(\beta)}{2} + \frac{L^2M\Omega^2\cos(\beta)\sin(\beta)}{4}\right) - \\
        &\frac{L^2M\Omega^2\cos(\beta)^2\cos(\phi)\sin(\phi)}{6} = 0
    \end{split}
\end{equation}

\subsection{}
Equation of motion for the case that the rotation around the vertical bar is not constant:

\begin{equation}
    \begin{split}
        &\frac{L^2\mathrm{\ddot{\theta}}\left(16M+16m+12M\cos\left(\beta \right)+12m\cos\left(\beta \right)+2M{\cos\left(\beta \right)}^2+4m{\cos\left(\beta \right)}^2-2M{\cos\left(\beta \right)}^2{\cos\left(\phi \right)}^2\right)}{12}+\\
        &\frac{L^2\mathrm{\ddot{\theta}}\left(12M\cos\left(\phi \right)\sin\left(\beta \right)+6M\cos\left(\beta \right)\cos\left(\phi \right)\sin\left(\beta \right)\right)}{12}
    \end{split}
\end{equation}


\subsection{}
% \begin{equation}
%     \begin{split}
%         &-c{\left(L\mathrm{\dot{\theta}}+\mathrm{\dot{\theta}}x_{2}\cos\left(\beta \right)-\mathrm{\dot{\phi}}x_{1}\cos\left(\phi \right)+L\mathrm{\dot{\theta}}\cos\left(\beta \right)-\mathrm{\dot{\theta}}x_{1}\cos\left(\phi \right)\sin\left(\beta \right)\right)}^2-\\
%         &c{x_{1}}^2{\sin\left(\phi \right)}^2{\left(\mathrm{\dot{\theta}}+\mathrm{\dot{\phi}}\sin\left(\beta \right)\right)}^2-c{\mathrm{\dot{\phi}}}^2{x_{1}}^2{\cos\left(\beta \right)}^2{\sin\left(\phi \right)}^2 
%     \end{split}
% \end{equation}

% To find the whole work done by this force distribution we have to integrate it:

% $x_1$ goes from -L to 0 and $x_2$ goes from -L/2 to L/2.

% \begin{equation}
%     \begin{split}
%         &-c\int_{-L}^0\int_{-\frac{L}{2}}^{\frac{L}{2}}
%         \underbrace{\left[{\left(L\mathrm{\dot{\theta}}+\mathrm{\dot{\theta}}x_{2}\cos\left(\beta \right)-\mathrm{\dot{\phi}}x_{1}\cos\left(\phi \right)+L\mathrm{\dot{\theta}}\cos\left(\beta \right)-\mathrm{\dot{\theta}}x_{1}\cos\left(\phi \right)\sin\left(\beta \right)\right)}^2\right]}_{\text{A}}
%         dx_2dx_1\\
%         &-c\int_{-L}^0\int_{-\frac{L}{2}}^{\frac{L}{2}}
%         \underbrace{\left[{x_{1}}^2\left({\sin\left(\phi \right)}^2{\left(\mathrm{\dot{\theta}}+\mathrm{\dot{\phi}}\sin\left(\beta \right)\right)}^2+{\mathrm{\dot{\phi}}}^2{\cos\left(\beta \right)}^2{\sin\left(\phi \right)}^2\right)\right]}_{\text{B}}
%          dx_2dx_1
%     \end{split}
% \end{equation}

% Starting with the integration of A:



% \begin{equation}
%     \begin{split}
%         &-c\int_{-L}^0
%         \left[\frac{1}{3\dot\theta\cos(\beta)}{\left(L\mathrm{\dot{\theta}}+\mathrm{\dot{\theta}}x_{2}\cos\left(\beta \right)-\mathrm{\dot{\phi}}x_{1}\cos\left(\phi \right)+L\mathrm{\dot{\theta}}\cos\left(\beta \right)-\mathrm{\dot{\theta}}x_{1}\cos\left(\phi \right)\sin\left(\beta \right)\right)}^3\right]
%         _{-\frac{L}{2}}^\frac{L}{2}dx_1 =
%     \end{split}
% \end{equation}

% \begin{equation}
%     \begin{split}
%         &-c\int_{-L}^0
%         \left[\frac{1}{3\dot\theta\cos(\beta)}{\left(L\mathrm{\dot{\theta}}+\mathrm{\dot{\theta}}\frac{L}{2}\cos\left(\beta \right)-\mathrm{\dot{\phi}}x_{1}\cos\left(\phi \right)+L\mathrm{\dot{\theta}}\cos\left(\beta \right)-\mathrm{\dot{\theta}}x_{1}\cos\left(\phi \right)\sin\left(\beta \right)\right)}^3\right]
%         dx_1\\
%         & +c\int_{-L}^0
%         \left[\frac{1}{3\dot\theta\cos(\beta)}{\left(L\mathrm{\dot{\theta}}-\mathrm{\dot{\theta}}\frac{L}{2}\cos\left(\beta \right)-\mathrm{\dot{\phi}}x_{1}\cos\left(\phi \right)+L\mathrm{\dot{\theta}}\cos\left(\beta \right)-\mathrm{\dot{\theta}}x_{1}\cos\left(\phi \right)\sin\left(\beta \right)\right)}^3\right]
%         dx_1
%     \end{split}
% \end{equation}

% Which can be summarized as

% \begin{equation}
%     \begin{split}
%         &-\frac{c}{3\dot\theta\cos(\beta)}\int_{-L}^0
%         \left(L\dot\theta-x_{1}\cos\left(\phi \right)\left(\mathrm{\dot\phi}+\mathrm{\dot\theta\sin(\beta)}\right)
%         +\frac{L\dot\theta\cos\left(\beta \right)}{2}\right)^3\\
%         &-\left(L\dot\theta-x_{1}\cos\left(\phi \right)\left(\mathrm{\dot\phi}+\mathrm{\dot\theta\sin(\beta)}\right)+\frac{3L\dot\theta\cos\left(\beta \right)}{2}\right)^3dx_1 =
%     \end{split}
% \end{equation}


% \begin{equation}
%     \begin{split}
%         \frac{c}{3\dot\theta\cos(\beta)}\frac{1}{\cos\left(\phi \right)\left(\mathrm{\dot\phi}+\mathrm{\dot\theta\sin(\beta)}\right)}&
%         \left(\left[\left(L\dot\theta-x_{1}\cos\left(\phi \right)\left(\mathrm{\dot\phi}+\mathrm{\dot\theta\sin(\beta)}\right)
%         +\frac{L\dot\theta\cos\left(\beta \right)}{2}\right)^4\right]_{-L}^0\right.\\
%         &\left.-\left[\left(L\dot\theta-x_{1}\cos\left(\phi \right)\left(\mathrm{\dot\phi}+\mathrm{\dot\theta\sin(\beta)}\right)+\frac{3L\dot\theta\cos\left(\beta \right)}{2}\right)^4\right]_{-L}^0\right) =
%     \end{split}
% \end{equation}

% Which finally yields:
% \begin{equation}
%     \begin{split}
%         \frac{c}{3\dot\theta\cos(\beta)}*\frac{1}{\cos\left(\phi \right)\left(\mathrm{\dot\phi}+\mathrm{\dot\theta\sin(\beta)}\right)}*&\left(\left(L\dot\theta+\frac{L\dot\theta\cos\left(\beta \right)}{2}\right)^4\right.\\
%         &-\left(L\dot\theta+\frac{3L\dot\theta\cos\left(\beta \right)}{2}\right)^4 -\\
%         &-\left(L\dot\theta+L\cos\left(\phi \right)\left(\mathrm{\dot\phi}+\mathrm{\dot\theta\sin(\beta)}\right)
%         +\frac{L\dot\theta\cos\left(\beta \right)}{2}\right)^4\\
%         &\left.+\left(L\dot\theta+L\cos\left(\phi \right)\left(\mathrm{\dot\phi}+\mathrm{\dot\theta\sin(\beta)}\right)+\frac{3L\dot\theta\cos\left(\beta \right)}{2}\right)^4\right)
%     \end{split}
% \end{equation}

% Which can be simplified again to:

% \begin{equation}
%     \begin{split}
%         \frac{c}{3\dot\theta\cos(\beta)}*\frac{1}{\cos\left(\phi \right)\left(\mathrm{\dot\phi}+\mathrm{\dot\theta\sin(\beta)}\right)}*&\left({\left(L\dot\theta+\frac{3L\dot\theta\cos\left(\beta \right)}{2}+L\cos\left(\phi \right)\left(\mathrm{\dot\phi}+\dot\theta\sin\left(\beta \right)\right)\right)}^4-\right.\\
%         &\left(L\dot\theta+\frac{L\dot\theta\cos\left(\beta \right)}{2}+L\cos\left(\phi \right)\left(\mathrm{\dot\phi}+\dot\theta\sin\left(\beta \right)\right)\right)^4+\\
%         &\left(L\dot\theta+\frac{L\dot\theta\cos\left(\beta \right)}{2}\right)^4-\\
%         &\left.\left(L\dot\theta+\frac{3L\dot\theta\cos\left(\beta \right)}{2}\right)^4\right) = A
%     \end{split}
% \end{equation}


% Analogous for B:

% \begin{equation}
%     \begin{split}
%         &-c\int_{-L}^0\int_{-\frac{L}{2}}^{\frac{L}{2}}
%         \underbrace{\left[{x_{1}}^2\left({\sin\left(\phi \right)}^2{\left(\mathrm{\dot{\theta}}+\mathrm{\dot{\phi}}\sin\left(\beta \right)\right)}^2+{\mathrm{\dot{\phi}}}^2{\cos\left(\beta \right)}^2{\sin\left(\phi \right)}^2\right)\right]dx_2dx_1}_{\text{B}} \overset{B \neq B(x_2)}{=} \\
%         &-cL\int_{-L}^0
%         \left[{x_{1}}^2\left({\sin\left(\phi \right)}^2{\left(\mathrm{\dot{\theta}}+\mathrm{\dot{\phi}}\sin\left(\beta \right)\right)}^2+{\mathrm{\dot{\phi}}}^2{\cos\left(\beta \right)}^2{\sin\left(\phi \right)}^2\right)\right]dx_1 = \\
%         &-\frac{cL}{3}\left({\sin\left(\phi \right)}^2{\left(\mathrm{\dot{\theta}}+\mathrm{\dot{\phi}}\sin\left(\beta \right)\right)}^2+{\mathrm{\dot{\phi}}}^2{\cos\left(\beta \right)}^2{\sin\left(\phi \right)}^2\right)\left[x_1^3\right]_{-L}^0 = \\
%         &-\frac{cL^4}{3}\left({\sin\left(\phi \right)}^2{\left(\mathrm{\dot{\theta}}+\mathrm{\dot{\phi}}\sin\left(\beta \right)\right)}^2+{\mathrm{\dot{\phi}}}^2{\cos\left(\beta \right)}^2{\sin\left(\phi \right)}^2\right)
%     \end{split}
% \end{equation}

% And to bring the ring to Mordor:

% $A + B = W =$

% \begin{equation}
%     \begin{split}
%         &-\frac{c}{3\dot\theta{\cos\left(\beta \right)}^2\left(\mathrm{\dot\phi}+\dot\theta\sin\left(\beta \right)\right)}*\left(\left(L\dot\theta+\frac{L\dot\theta\cos\left(\beta \right)}{2}+L\cos\left(\phi \right)\left(\mathrm{\dot\phi}+\dot\theta\sin\left(\beta \right)\right)\right)^4\right.\\
%         &\left.-{\left(L\dot\theta+\frac{3L\dot\theta\cos\left(\beta \right)}{2}+L\cos\left(\phi \right)\left(\mathrm{\dot\phi}+\dot\theta\sin\left(\beta \right)\right)\right)}^4-{\left(L\dot\theta+\frac{L\dot\theta\cos\left(\beta \right)}{2}\right)}^4+{\left(L\dot\theta+\frac{3L\dot\theta\cos\left(\beta \right)}{2}\right)}^4\right)
%     \end{split}
% \end{equation}

After matlab integration of the work per area we get for the work

\begin{equation}
    \begin{split}
        &W = \\
        &\frac{L^4\left(c{\left(\dot\theta+\mathrm{\dot\phi}\sin\left(\beta \right)\right)}^2\left({\cos\left(\phi \right)}^2-1\right)-c{\left(\mathrm{\dot\phi}\cos\left(\phi \right)+\dot\theta\cos\left(\phi \right)\sin\left(\beta \right)\right)}^2+c{\mathrm{\dot\phi}}^2{\cos\left(\beta \right)}^2\left({\cos\left(\phi \right)}^2-1\right)\right)}{3}\\
        &-L\left(Lc{\left(L\dot\theta+L\dot\theta\cos\left(\beta \right)\right)}^2+\frac{L^3c{\dot\theta}^2{\cos\left(\beta \right)}^2}{12}\right)-L^3c\left(L\dot\theta+L\dot\theta\cos\left(\beta \right)\right)\left(\mathrm{\dot\phi}\cos\left(\phi \right)+\dot\theta\cos\left(\phi \right)\sin\left(\beta \right)\right)
    \end{split}
\end{equation}

Which results in the equations of motion:

\begin{equation}
    \begin{split}
        &\frac{L^2M\mathrm{\ddot\phi}}{3}+\frac{2L^4c\left(\mathrm{\dot\phi}+\dot\theta\sin\left(\beta \right)\right)}{3}+\frac{L^2m\mathrm{\ddot\phi}}{3}-\frac{L^2m\mathrm{\ddot\phi}{\cos\left(\beta \right)}^2}{3}+\frac{LMg\cos\left(\beta \right)\sin\left(\phi \right)}{2}+\\
        &L^4c\dot\theta\cos\left(\phi \right)\left(\cos\left(\beta \right)+1\right)+\frac{L^2M{\dot\theta}^2\sin\left(\beta \right)\sin\left(\phi \right)}{2}-\frac{L^2M{\dot\theta}^2{\cos\left(\beta \right)}^2\cos\left(\phi \right)\sin\left(\phi \right)}{6}+\\
        &\frac{L^2M{\dot\theta}^2\cos\left(\beta \right)\sin\left(\beta \right)\sin\left(\phi \right)}{4}\hspace{25mm} \text{for } \phi
    \end{split}
\end{equation}

And 

\begin{equation}
    \begin{split}
        &\frac{cL^4}{6}\left(16\dot\theta+24\dot\theta\cos\left(\beta \right)+6\mathrm{\dot\phi}\cos\left(\phi \right)+4\mathrm{\dot\phi}\sin\left(\beta \right)+13\dot\theta{\cos\left(\beta \right)}^2-4\dot\theta{\cos\left(\beta \right)}^2{\cos\left(\phi \right)}^2+6\mathrm{\dot\phi}\cos\left(\beta \right)\cos\left(\phi \right)+12\dot\theta\cos\left(\phi \right)\sin\left(\beta \right)+\right.\\
        &\left.12\dot\theta\cos\left(\beta \right)\cos\left(\phi \right)\sin\left(\beta \right)\right)+\\
        &\frac{\mathrm{\ddot\theta}}{12} \left(16M+16m+12M\cos\left(\beta \right)+12m\cos\left(\beta \right)+2M{\cos\left(\beta \right)}^2+4m{\cos\left(\beta \right)}^2-2M{\cos\left(\beta \right)}^2{\cos\left(\phi \right)}^2+\right.\\
        &\left.12M\cos\left(\phi \right)\sin\left(\beta \right)+6M\cos\left(\beta \right)\cos\left(\phi \right)\sin\left(\beta \right)\vphantom{\int}\right)L^2\hspace{45mm}\text{for } \theta
    \end{split}
\end{equation}


\subsection{Plugging in the values:}

Let $L = 0.25 [m], \beta = \frac{\pi}{6},M= 0.5 [kg],m= 0.2 [kg], g = 9.81[\frac{m}{s^2}] \text{ and } c = 0.1[\frac{Ns}{m^3}]$.\vspace{3mm}\\

The equations become:

\begin{equation}
    \begin{split}
        &\frac{\mathrm{\dot\phi}}{3840}+\frac{11\mathrm{\ddot\phi}}{960}+\frac{\dot\theta}{7680}+\frac{{\dot\theta}^2\sin\left(\phi \right)}{128}+\frac{981\sqrt{3}\sin\left(\phi \right)}{3200}+\frac{\sqrt{3}{\dot\theta}^2\sin\left(\phi \right)}{512}+\frac{\dot\theta\cos\left(\phi \right)\left(\frac{\sqrt{3}}{2}+1\right)}{2560}\\
        &-\frac{{\dot\theta}^2\cos\left(\phi \right)\sin\left(\phi \right)}{256}\hspace{25mm} \text{for } \phi
    \end{split}
\end{equation}

And

\begin{equation}
    \begin{split}
        &\frac{\mathrm{\dot\phi}}{7680}+\frac{103\dot\theta}{61440}-\frac{\dot\theta{\cos\left(\phi \right)}^2}{5120}+\frac{\sqrt{3}\dot\theta}{1280}+\\
        &\frac{\mathrm{\ddot\theta}}{192}\left(3\cos\left(\phi \right)-\frac{3{\cos\left(\phi \right)}^2}{4}+\frac{21\sqrt{3}}{5}+\frac{3\sqrt{3}\cos\left(\phi \right)}{4}+\frac{251}{20}\right)+\\
        &\frac{\mathrm{\dot\phi}\cos\left(\phi \right)}{2560}+\frac{\dot\theta\cos\left(\phi \right)}{2560}+\frac{\sqrt{3}\mathrm{\dot\phi}\cos\left(\phi \right)}{5120}+\frac{\sqrt{3}\dot\theta\cos\left(\phi \right)}{5120}\hspace{45mm}\text{for } \theta
    \end{split}
\end{equation}

\subsection{Equilibrium State}

Reminder- The potential energy looks as follows:

\begin{equation}
    \begin{split}
        \frac{LMg\left(2\sin\left(\beta \right)-\cos\left(\beta \right)\cos\left(\phi \right)\right)}{2}
    \end{split}
\end{equation}
\textbf{Case 1: $\Omega$ fixed:} \vspace{0.5cm}\\

The derivative of the potential energy w.r.t. $\theta$ is always 0.

\begin{equation}
    \begin{split}
        \frac{\partial V}{\partial \theta} = 0
    \end{split}
\end{equation}

The derivative of the potential energy w.r.t. $\phi$ not however:

\begin{equation}
    \begin{split}
        \frac{\partial V}{\partial \phi} = \frac{LMg\cos\left(\beta \right)\sin\left(\phi \right)}{2}
    \end{split}
\end{equation}

\noindent Equation 12 is only 0 for $\boldsymbol{\phi = k*\pi}$ which makes sense as in this configuration the square has a momentary velocity in horizontal direction which doesn't change the altitude of any body.\vspace{0.5cm}\\

\textbf{Case 2: $\dot\theta$ can change:} \vspace{0.3cm}\\

As V is independent of $\theta$ altogether the result is the same for this case.\vspace{3mm}\\

\noindent \textbf{The equation of motions are after plugging in equilibrium state:}\vspace{3mm}\\

% Plugging in $\phi = \dot\phi = \ddot\phi = 0$:

\begin{equation}
    \begin{split}
        &\phi:\\
        &0 = 0
    \end{split}
\end{equation}

\begin{equation}
    \begin{split}
        &\theta:\\
        &\frac{L^2\mathrm{\ddot\theta}}{12}\left(16M+16m+12M\cos\left(\beta \right)+12M\sin\left(\beta \right)+12m\cos\left(\beta \right)+\right.\\
        &\left.4m{\cos\left(\beta \right)}^2+6M\cos\left(\beta \right)\sin\left(\beta \right)\vphantom{\int_1^2}\right) = 0
    \end{split}
\end{equation}
As the coefficient of $\ddot\theta$ is constant, the angular acceleration of the vertical shaft has to be 0. This results in a constant angular velocity which recovers the case of a constant $\Omega$.



    


